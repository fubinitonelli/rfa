%!TEX root = ../main.tex

%\usepackage[a4paper, twoside, left=3cm, right=2cm, top=3cm, bottom=2.5cm, showframe]{geometry} %A4
\usepackage[
	paperwidth=17cm,
	paperheight=24cm,
	twoside,
	left=2.5cm,
	right=2cm,
	top=2.5cm,
	bottom=3cm,
	%showframe
]{geometry} %17 X 24

\usepackage{fancyhdr}

% https://tex.stackexchange.com/questions/286541/how-to-properly-add-appendices-to-the-toc
\usepackage[page,toc,titletoc,title]{appendix}

\usepackage{setspace}

\usepackage{ifthen}

\usepackage{amsmath}
\allowdisplaybreaks

\usepackage[T1]{fontenc} % Use 8-bit encoding that has 256 glyphs
\usepackage[utf8]{inputenc} % Required for including letters with accents

\usepackage[english]{babel}
\usepackage[colorlinks=true]{hyperref}

\AtBeginDocument{
	\ifthenelse{\boolean{stampa}}{
		\hypersetup{linkcolor=black,urlcolor=black}
	}{
		\hypersetup{linkcolor=blue,urlcolor=blue}
	}
}

\usepackage[nospace]{varioref} % after the babel package! [https://ctan.mirror.garr.it/mirrors/ctan/macros/latex/required/tools/varioref.pdf]

\usepackage{placeins} % allows for using the \FloatBarrier
\usepackage[small,sf,bf]{titlesec}
\usepackage[parfill]{parskip}

\usepackage{cancel}

\usepackage{amsthm}
\usepackage{amssymb}
\usepackage{bookmark}
\usepackage{mathtools}
\usepackage{accents}
\usepackage{centernot}
\usepackage{dsfont} %per indicatrice
\usepackage[scr=boondox]{mathalfa}
\usepackage{marginnote}
\usepackage[textsize=tiny, textwidth=1.5cm]{todonotes} %add disable to options to not show in pdf
\usepackage{xcolor}
\usepackage{pgfplots}
\usepackage{float}
\usepackage{fp}
\usepackage{nicefrac}
\usepackage{enumitem}
\usepackage{pdfpages}
\pgfplotsset{compat=1.14}
\usetikzlibrary{hobby,arrows,patterns,calc,intersections,arrows.meta}

\newcommand\DrawBlock[3]{
\ifx#1b\relax
  \path[draw]
    (lm\the\numexpr#2-1\relax) -- ++(0,0,#3) coordinate (blocklf)
    (bm\the\numexpr#2-1\relax) -- ++(0,0,#3) coordinate (blocklb)
    (lm#2) -- ++(0,0,#3) coordinate (blockrf)
    (bm#2) -- ++(0,0,#3) coordinate (blockrb);
  \filldraw[fill=white,draw=black]
    (lm\the\numexpr#2-1\relax) -- (blocklf) -- (blocklb) -- (blockrb) -- (blockrf) -- (lm#2);
\else  
  \ifx#1f\relax
    \path[draw]
      (fm\the\numexpr#2-1\relax) -- ++(0,0,#3) coordinate (blocklf)
      (lm\the\numexpr#2-1\relax) -- ++(0,0,#3) coordinate (blocklb)
      (fm#2) -- ++(0,0,#3) coordinate (blockrf)
      (lm#2) -- ++(0,0,#3) coordinate (blockrb);
    \filldraw[fill=white,draw=black]
      (fm\the\numexpr#2-1\relax) -- (blocklf) -- (blocklb) -- (blockrb) -- (blockrf) -- (fm#2);
  \fi
\fi
\draw (blocklf) -- (blockrf);
}

% Load widebar without loading the entire mathabx
\DeclareFontFamily{U}{mathx}{\hyphenchar\font45}
\DeclareFontShape{U}{mathx}{m}{n}{<-> mathx10}{}
\DeclareSymbolFont{mathx}{U}{mathx}{m}{n}
\DeclareMathAccent{\widebar}{0}{mathx}{"73}

%aggiunti
\usepackage[footnoteinside = false]{mdframed} %per box teoremi
\usepackage{footnote} %testing GG
\usepackage{tabto}
\usepackage{subcaption}
