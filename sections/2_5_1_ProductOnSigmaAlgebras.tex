%!TEX root = ../main.tex
\subsubsection{Product measures}
\todo{part of this can be taken as intro of the subsection.}
In this section we discuss on how to measure a function which domain is not mono-dimensional.
We know yet how the Riemann integration has solved this problem:

%\missingfigure{Here we should put an image of Riemann in $\RR^2$ measure with axis and a proper patatoide.}
\begin{figure}[htpb]
	\centering
	\begin{tikzpicture}[y={(0:1cm)},x={(225:0.86cm)}, z={(90:1cm)},yscale=0.7,xscale=0.7]

	% coordinates for the lower grid
	\path
	  (1,3,0) coordinate (bm0) -- 
	  (4,3,0) coordinate (fm0) coordinate[midway] (lm0) --
	  (4,8,0) coordinate[pos=0.25] (fm1) coordinate[midway] (fm2) coordinate[pos=0.75] (fm3) coordinate (fm4) --
	  (1,8,0) coordinate (bm4) coordinate[midway] (lm4)--
	  (bm0) coordinate[pos=0.25] (bm3) coordinate[midway] (bm2) coordinate[pos=0.75] (bm1);
	\draw[dashed]
	  (lm0) -- 
	  (lm4) coordinate[pos=0.25] (lm1) coordinate[midway] (lm2) coordinate[pos=0.75] (lm3);

	% the blocks
	\DrawBlock{b}{1}{4}
	\DrawBlock{b}{2}{3.7}
	\DrawBlock{b}{3}{4.3}
	\DrawBlock{b}{4}{5}
	\DrawBlock{f}{1}{3.3}
	\DrawBlock{f}{2}{3.5}
	\DrawBlock{f}{3}{4}
	\DrawBlock{f}{4}{4.7}

	\foreach \Point/\Height in {lm1/3.7,lm2/4.3,lm3/5}
	  \draw[ultra thin,dashed,opacity=0.2] (\Point) -- ++(0,0,\Height);

	% the lower grid
	\foreach \x in {1,2,3}
	  \draw[dashed] (fm\x) -- (bm\x);
	\draw[dashed] (fm0) -- (bm0) -- (bm4);
	\draw (fm0) -- (fm4) -- (bm4);
	\draw[dashed] (lm0) -- (lm4);

	% coordinates for the surface
	\coordinate (curvefm0) at ( $ (fm0) + (0,0,4) $ );
	\coordinate (curvebm0) at ( $ (bm0) + (0,0,4) $ );
	\coordinate (curvebm4) at ( $ (bm4) + (0,0,6) $ );
	\coordinate (curvefm4) at ( $ (fm4) + (0,0,5.7) $ );

	% the surface
	\filldraw[ultra thick,fill=gray!25,fill opacity=0.2]
	  (curvefm0) to[out=-30,in=210] 
	  (curvefm4) to[out=-4,in=260]
	  (curvebm4) to[out=215,in=330]
	  (curvebm0) to[out=240,in=-20]
	  (curvefm0);

	% lines from grid to surface
	\draw[very thick,name path=leftline] (curvefm0) -- (fm0);
	\draw[very thick] (curvefm4) -- (fm4);
	\draw[very thick,name path=rightline] (curvebm4) -- (bm4);
	\draw[very thick,dashed] (curvebm0) -- (bm0);

	% coordinate system
	\coordinate (O) at (0,0,0);
	\draw[-latex] (O) -- +(5,0,0) node[above left] {$x$};
	\path[name path=yaxis] (O) -- +(0,10,0) coordinate (yaxisfinal) node[above] {$y$};
	\draw[-latex] (O) -- +(0,0,5) node[left] {$z$};
	\path[name intersections={of=yaxis and leftline,by={yaxis1}}];
	\path[name intersections={of=yaxis and rightline,by={yaxis2}}];
	\draw (O) -- (yaxis1);
	\draw[densely dashed,opacity=0.1] (yaxis1) -- (yaxis2);
	\draw[-latex] (yaxis2) -- (yaxisfinal);

	% for debugging
	%\foreach \Name in {bm0,fm0,lm0,fm1,fm2,fm3,fm4,bm4,lm4,bm1,bm2,bm3,lm1,lm2,lm3,%
	%curvefm0,curvebm0,curvebm4,curvefm4}
	%  \node at (\Name) {\Name};  
	\end{tikzpicture}
\end{figure}
\FloatBarrier

The main idea is to calculate the integral through what we know in one dimension, using the techniques for the integration of one variable:
$$\iint_\Omega f(x,y) \dx \dy=\int_a^b \left( \int_{\Omega_x} f(x,y) \dy \right) \dx=\int_c^d \left( \int_{\Omega_y} f(x,y) \dx \right) \dy$$

Now we need to extend this to the Lebesgue and abstract integral. This technique should be extended to proceed for a very general set. Our goal in this part is to extend the ``reduction'' we have just explained to the Lebesgue integral or, better, to the abstract integral.

\paragraph{Product of $\sigma$-algebras} Our first step is to build a proper $\sigma$-algebra. Let $\left( X, \Mg \right), \; \left(Y, \Ng\right)$ measure spaces, with $X,Y \neq \varnothing$.\\
How can we construct a product $\sigma$-algebra $\Mg\otimes\Ng$ such that the ``restrictions'' of its elements with respect to $X$ (or $Y$) are measurable set in $\Ng$ (or $\Mg$)?

\begin{defn}
	A set $\Rc \in \Pc (X \times Y)$ is a \emph{measurable rectangle} if:
	$$\Rc = A \times B$$
	where $A\in \Mg$ and $B \in \Ng$.
\end{defn}

\begin{defn}
	Let $\Mg $ and $\Ng$ be two $\sigma$-algebra. We define the \emph{product $\sigma$-algebra} as 
	$$\Mg \otimes \Ng \coloneqq \sca{\{A \times B : A \in \Mg, B \in \Ng \}}.$$
	This is the $\sigma$-algebra generated by all the measurable rectangles.
\end{defn}
This is the smallest $\sigma$-algebra containing all the measurable rectangles.
At this point you should be able to prove that $ \Bc(\RR^2) = \Bc(\RR) \otimes \Bc(\RR)$

The symbol $\otimes$ is known as tensor product but we will not study in deep this operation.

\begin{defn}
	Let $E \subset X \times Y$. Then:
	
	$E_x \coloneqq \{y \in Y: (x,y) \in E\} \subset Y$ is called the \emph{$x$-section} of $E$;
	
	$E_y \coloneqq \{x \in X: (x,y) \in E\} \subset X$ is called the \emph{$y$-section} of $E$.
\end{defn}

Notice that, if we have a rectangle $E=A \times B$, then we have $E_x = B$ for all $x \in A$ and  $E_y = A$ for all $y \in B$.\\
Through the last definition we can provide a proper characterization of the product $\sigma$-algebra $\Mg \otimes \Ng$:

\begin{prop}[characterization of the product $\sigma$-algebra] \label{prop-charact-Mg-x-Ng}
	Consider a generic rectangle $E \subset X \times Y$.\\
	We can define $\Mg \otimes \Ng$ as follows:
	\begin{align*}
		\Mg \otimes \Ng
		&=\{E \in \Mg \otimes \Ng \text{ such that } E_x \in \Ng\quad \forall x \in X\} \\
		&=\{E \in \Mg \otimes \Ng \text{ such that }E_y \in \Mg\quad \forall y \in Y\}
	\end{align*}
\end{prop}

\begin{proof}
	Set 
	$$\Cc \coloneqq \{E\in \Mg \otimes \Ng :\  E_x \in \Ng \  \forall x \in X\}.$$
	Obviously $\Cc \subseteq  \Mg \otimes \Ng$, and any measurable rectangle is in $\Cc$. To prove the opposite relation, we just need to prove that $\Cc$ is a $\sigma$-algebra as $\Mg \otimes \Ng$ is the generated minimal $\sigma$-algebra.

	First, it is easy to see that any measurable rectangle belongs to $\Cc$, namely for any $A \in \Mg$ and $B \in \Ng$, $A \times B \in \Cc$. In particular, $X \times Y \in \Cc$.
	
	Then, notice that if $E \in \Cc$ then $(E\comp)_x =(E_x)\comp \in \mm$, which in turn implies $E\comp \in \Cc$.

	Finally, consider $E = \bigcup_{n \in \NN} E_n$ with $E_n \in \Cc$. Then:
	$$E_x = \left( \bigcup_{n \in \NN} E_n \right)_{X}
	= \bigcup_{n \in \NN} (E_n)_x$$
	The same goes for y-sections.
\end{proof}

There may exist a set $E$ such that $E_x \in \Ng$ for all $x \in X$ and $E_y \in \Mg$ for all $y \in Y$ but $E \notin \Mg \otimes \Ng$.\footnote{For further discussion, see: E. Hewitt, K. Stromberg, Real and Abstract Analysis, page 393, exercise 21.26.}

\paragraph{Measurable functions} Now that we have a proper structure we can talk about measurability.

\begin{defn}
	Let $f: X \times Y \to \RR$, or $[0,+\infty]$, be defined everywhere.\\
	Then we define the \emph{x-section $f^x$} and the \emph{y-section $f^y$} of $f$ by setting:
	$$f^x:Y\to \RR,\quad f^x(y) \coloneqq f(x,y)\quad \forall x \in X \text{ fixed};$$
	$$f^y:X\to \RR,\quad f^y(x) \coloneqq f(x,y)\quad \forall y \in Y \text{ fixed}.$$
	(Pay attention: those symbols are not partial derivatives.)
\end{defn}
Notice that it's superfluous now to say ``everywhere'' because there is still no measure.

\begin{prop}
	If $f:(X \times Y, \Mg \otimes \Ng) \to \RR$, or $[0, +\infty]$ is $(\Mg \otimes \Ng)$-measurable, then:
	$$f^x\text{ is }\Ng\text{-measurable}\quad \forall x \in X;$$
	$$f^y\text{ is }\Mg\text{-measurable}\quad \forall y \in Y.$$ 
\end{prop}
\begin{proof}
	Let $A \subseteq f(X \times Y)$ be open. Then:
	$$E = f^{-1}(A)=\{(x,y) \in X\times Y: f(x,y) \in A\} \in \Mg \otimes \Ng.$$
	Therefore for any $A$ we have (see characterization of product $\sigma$-algebra, proposition \vref{prop-charact-Mg-x-Ng}):
	\begin{alignat*}{2}
		E_x = (f^x)^{-1}(A) &= (f^x)^{-1}(A)\ \in \Ng \quad &&\forall x \in X, \\
		E_y = (f^y)^{-1}(A) &= (f^y)^{-1}(A)\ \in \Mg \quad &&\forall y \in Y.
	\end{alignat*}
	This means that $f^x$ and $f^y$ are measurable with respect to the corresponding $\sigma$-algebras.
\end{proof}


\paragraph{Product measure} Now our setting is quite ready: the last theoretical concept that has to be extended is the measure itself. Consider the measure spaces $(X,\Mg, \mu)$, $(Y,\Ng, \nu)$. We would like to build a new measure space with $X \times Y$ and $\Mg \otimes \Ng$. Which \textit{measure} could be suitable?

First, take into account the following result.
\begin{prop} \label{fin-meas-cart-integrable}%prop3
	Let $\mu$, $\nu$ two $\sigma$-finite measures, and $E\in \Mg \otimes \Ng$. Then:
	\begin{itemize}
		\item $ x \to \nu(E_x)$ is $\Mg$-measurable;
		\item $y \mapsto \mu(E_y)$ is $\Ng$-measurable;
		\item $\int_x \nu(E_x) \,\dmu = \int_y \mu (E_y) \,\dnu$.
	\end{itemize}
\end{prop}\footnote{To see the proof: A. N. Kolmogorov, S. V. Fomin, Introductory Real Analysis, 1975, pages 356-359, theorem 3.}

The previous proposition suggests the following.
\begin{defn}
	Consider the set function $\mu \otimes \nu : \Mg \otimes \Ng \to [0,+\infty]$ defined as follows:
	$$
	(\mu \otimes \nu)(E) 
	\coloneqq  \int_X\nu(E_x) \dmu 
	= \int_Y \mu (E_y) \dnu 
	\quad \forall E \in \Mg \otimes \Ng
	.
	$$
	Then $\mu \otimes \nu$ is a $\sigma$-finite measure and is called \emph{product measure} of $\mu$ and $\nu$.
\end{defn}

Observe that for any measurable rectangle 
$$
R
=A\times B 
\in \Mg \otimes \Ng
$$
we have
$$
(\mu \otimes \nu)(R)
=\mu(A) \cdot \nu(B)
.
$$

In general, $\mu \otimes \nu$ is not complete even if $\mu$ and $\nu$ are so (see definition \vref{defn-complete-measure}).
\begin{exam}
	Take the sets $X=Y=[0,1]$ with $\sigma$-algebras $\Mg=\Ng=\Lc([0,1])$ and measure $\mu = \nu = \lambda$. \\
	Then let $A\in \Lc ([0,1])$ such that $\lambda(A)=0$, and $B \in \Lc ([0,1])$ such that $\lambda(B)>0$.
	So we have $(\lambda \otimes \lambda)(A \times B)=0$. \\
	Now consider the Vitali set $U \subset B$:  $A \times U \subset A \times B$ so $A \times U \notin \Lc([0,1]) \otimes \Lc([0,1])$.\\
	So $A\times U$ is zero-measure subset of the $\sigma$-algebra but it is not measurable then, by definition the measure $\lambda \otimes \lambda$ is not complete.
\end{exam}

