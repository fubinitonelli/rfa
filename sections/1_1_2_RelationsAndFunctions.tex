%!TEX root = ../main.tex
\subsubsection{Relations and functions}

Now introduce the concept of functions, starting from the more general notion of relation.
\begin{defn}
	Consider two sets $X,\,Y \neq \varnothing $.\\
	A \emph{binary relation} $r$ from $X$ to $Y$ is a subset of $X \times Y$.\\
	The \emph{domain} of a relation is the set of all elements of $X$ that belongs to the relation:
	$$
		\dom(r) 
		\coloneqq \{x \in X: (x,y) \in r \text{ for some } y \in Y\}
	.
	$$
	The \emph{range} of a relation is the set of all elements of $Y$ that belongs to the relation:
	$$ 
		\rng(r) 
		\coloneqq \{y \in Y: (x,y) \in r \text{ for some } x \in X\}
	.
	$$
\end{defn}
When $(x,y), (x,z) \in r \implies y = z$, then r is a single valued relation; this kind of relation plays an important role in mathematics, they are known also as functions:
\begin{defn}\label{function}
	A law (rule) $f:X\to Y$ is a \emph{function} (or mapping, map, transformation) if it associates to an element of $X$ one and only one element of $Y$.\\
	If $X = \NN$, then this function is called \emph{sequence}. 
\end{defn}

This definition was formulated by Dirichlet as follows:
\begin{quote}
	$y$ is a function of a variable $x$ defined on the interval $a<x<b$ if to every value of the variable $x$ in this interval there corresponds a definite value of the variable $y$.  Also, it is irrelevant in what way this correspondence is established.\footnotemark{}
\end{quote}
\footnotetext{Cited in I. Kleiner, Evolution of the Function Concept: A Brief Survey, College Mathematics Journal vol.\ 20 iss.\ 4, page 291.}

A function can be applied also to an entire set: consider for example $A \subset X$, then: 
$$
	f(A) 
	\coloneqq \{y \in Y: (x,y) \in f \text{ for some } x \in \dom(f)\cap A\}
.
$$

There exist different ways to write functions. If we want to highlight the relation to the elements instead, we will use one of those notation:
$$
	f:x\mapsto y, 
	\quad y=f\left(x\right)
$$
which are equivalent to $(x,y) \in f$, where $y$ is the \emph{image} of $x$ and $x$ is the \emph{preimage}\footnotemark{}
(or counterimage, inverse image)  of $y$.
\footnotetext{\itatrasl{controimmagine}}

In case we want to highlight how two set are related through a function $f$, then we will denote that as: $$f:X\to Y$$
In this case the set $Y$ image of the set $X$, while $X$ is the preimage of $Y$.

\begin{defn}
	The \emph{inverse function} of a function $f:X \to Y$ is denoted by $f^{-1}$ and, if $B \subseteq Y$, is defined as:
	$$
		f^{-1}(B) 
		= \{x \in X: x = f^{-1}(y) \text{ for some } y \in B\}
	.
	$$
\end{defn}


Here we state some properties with respect to union and intersection:
\begin{prop} [Set function properties]
	Consider the sets $A$, $A_j \subset X$ and $B_j \subset Y$ for $j=1,2$.\\
	Then:
	\begin{itemize}
		\item $f^{-1}(A\comp) = (f^{-1}(A))\comp$;
		\item $f\left(A_1\cup A_2\right)=f\left(A_1\right)\cup f\left(A_2\right) $;
		\item $f\left(A_1\cap A_2\right)\subseteq f\left(A_1\right)\cap f\left(A_2\right) $;
		\item $f^{-1}\left(B_1\cup B_2\right)=f^{-1}\left(B_1\right)\cup f^{-1}\left(B_2\right) $;
		\item $f^{-1}\left(B_1\cap B_2\right)=f^{-1}\left(B_1\right)\cap f^{-1}\left(B_2\right) $.
	\end{itemize}
\end{prop} 


Pay attention: the third property is not an equivalence like the other ones, but it states an inclusion. Let us give a counterexample.

\begin{exam}
	Let $f:\RR^2 \to \RR^2$, $f(x,y) \mapsto (x,0)$ be the function that projects on the $x$ axis. \\
	Let also $A=\{(x,0): \; x \in \left[0,1\right]\}$, and $B=\{(x,1): \;x \in \left[0,1\right]\}$.
	So we have $A\cap B= \varnothing$, and thus $f\left(A\cap B\right)=\varnothing$. \\
	But $f\left(B\right)=A=f\left(A\right)$, and finally $f\left(A\right)\cap f\left(B\right) =A$.
	\begin{figure}[htpb]
		\centering
		\begin{tikzpicture}[xscale=3,yscale=3]

			\draw[stealth-stealth] (0,1.5) -- (0,0) -- (1.5,0);

			\draw[ultra thick] (0,1) -- (1,1) node[midway, above] {$B$};
			\draw[ultra thick] (0,0) -- (1,0) node[midway, above] {$A$};

			\draw[dashed] (1,1) -- (1,0);

			\node[left] at (0,1) {$1$};
			\node[below left] at (0,0) {$0$};
			\node[below] at (1,0) {$1$};

		\end{tikzpicture}
	\end{figure}
	\FloatBarrier
\end{exam}

Functions can have different behavior with respect to how many times they relate each element of a set to another:

\begin{defn}
	A function $f: \dom(f) \to Y$ is:
	\begin{itemize}
		\item an \emph{injection} if $f(x) \neq f(\tilde{x})$ for $x\neq \tilde{x}$;
		\item a \emph{surjection} if $\rng(f) = Y$ or equivalently $f(\dom(f)) = Y$;
		\item a \emph{bijection} if it is both surjective and injective.
	\end{itemize}
\end{defn}

Now consider the case in which we have a function $f: X \to Y$ and a function $g:Y \to Z$. If every image of $f$ is in the domain of $g$ than we can build a new function from $X$ to $Z$ through a composition of $f$ and $g$:
\begin{defn}
	Let $f: X \to Y$ and $g: Y \to Z$ be two functions.\\
	Than the \emph{composition} of $g$ with $f$ is defined by:
	$$
		g \circ f 
		\coloneqq \{(x,z)\in X \times Z :\ \exists \, y\in Y :\ (x,y) \in f \wedge (y,z)\in g \}
	.
	$$
\end{defn}

It is also possible to define a binary relation from a function:
$$r(x,y) = f(x) - y, \quad r = \{(x,y) \in X \times Y : r(x,y) = 0\}.$$
In this case $(x,y) \in r$ if and only if $f(x)-y =0$, that is $r(x,y) = 0$.


\paragraph{Order relations}

We have seen that the relations are a generalization of functions, so functions are a sort of special kind of relations. There exists also other notably kind of relations, one of those are used to define an order inside a set:
\begin{defn}
	Let $X \neq \varnothing$, $r \subseteq X \times X$ is an \emph{order relation} in $X$ if the following properties holds (let $x,\ y,\ z \in X$):
	\begin{itemize}
		\item \emph{reflexivity}: $$\left(x,x\right)\in r;$$
		\item \emph{antisymmetry}: $$ \left( \left(x,y\right)\in r\wedge \left(y,x\right)\in r \right) \iff x=y;$$
		\item \emph{transitivity}: $$\left(\left(x,y\right)\in r \wedge \left(y,z\right)\in r\right)\implies \left(x,z\right)\in r.$$
	\end{itemize}
	A generic order relation is denoted by $\preceq$.
\end{defn}

A famous order relation is the canonical order in $\NN$ or in $\QQ$ or in $\RR$: this order is the principle that allow as to say that a number is greater to another. This particular relation is denoted by $\leq$. For example: $3 \leq 5$ or $7 \geq 2$. Observe that the strict ``version'' of the symbol ($<$) doesn't represent an order as it isn't reflexive.

With the canonical order we can always compare two number and say which is greater. This peculiarity doesn't belong to order relation in general:

\begin{defn}
	An order relation $r \subseteq X \times X$ is a \emph{total order} in $X$ if: 
	$$
		(x,y) \in r 
		\vee (y,x) \in r 
		\quad \forall x,y \in X
	,
	$$
	otherwise $r$ is a \emph{partial order}.
\end{defn} 

If exists a total order relation in a set $X$, then $X$ is a totally ordered set.

The inclusion between sets ($\subset$) is, for example, a partial order in $\Pc(X)$. Also the divisibility between two number in $\NN$, that is $n \preceq m$ if $n$ divides $m$ ($m$ is a multiple of $n$), is a partial order.


\paragraph{Equivalence relation} Like order relation, it exists another kind of binary relation with wide applications:
\begin{defn}
	Let $X \neq \varnothing$, $r \subseteq X \times X$ is an \emph{equivalence relation} in $X$ if the following properties holds (let $x,\ y,\ z \in X$):
	\begin{itemize}
		\item \emph{reflexivity}: $$\left(x,x\right)\in r;$$
		\item \emph{symmetry}: $$\left(x,y\right)\in r\iff\left(y,x\right)\in r;$$
		\item \emph{transitivity}: $$\left(\left(x,y\right)\in r \wedge \left(y,z\right)\in r\right)\implies \left(x,z\right)\in r.$$
	\end{itemize}
	Equivalence relations are denoted by the symbol $\sim$.
\end{defn}

Those properties are similar to the ones of the equality symbol ($=$), but while the equality state that two element are the same, an equivalent relation states that two elements are equivalent with respect to a certain property. Here some examples of equivalent relations:
\begin{exam}
	Consider the set of all straight lines in a given planes. Than the parallelism of two lines is an equivalent relation in such set.
\end{exam}		
\begin{exam}
	Fix $q \in \NN$, then 
	$$
		x\sim y 
		\iff \exists \, k \in \ZZ: x-y = kq
	$$
	is an equivalence relation in $\ZZ$.
\end{exam}		
\begin{exam} 
	The relation 
	$$
		(a,b) \sim (c,d) 
		\iff a+d = b+c
	$$
	is an equivalence relation in $\NN \times \NN$.
\end{exam}		
\begin{exam}
	The relation 
	$$
		(a,b) \sim (c,d) 
		\iff ad=bc
	$$
	is an equivalence relation in $\ZZ \times \ZZ_0$.
\end{exam}

To sum up with all this kind of relations, consider these two proposition $(x,y) \in r$ and $(y,x) \in r$. If $r$ is an order relation then both proposition can be false at the same time. If $r$ is a total order, the at least one of the two can be true. If $r$ is an equivalence relation, then they must be both true or both false.

\paragraph{Equivalence class} We said that an equivalence relation states that two elements are ``equivalent'' with respect to a certain property. Now consider if we need to dived the elements of a set with respect to this kind of equivalence. We would create a partition of our set where in each subset all the elements are equivalent:

\begin{defn}\label{defn-equiv-class-quot-set}
	Consider a set $X \neq \varnothing$ and an equivalence relation $\sim$  in $X \times X$.
	
	We define the \emph{equivalence class} of $x$ on $X$ with respect to $\sim$ as follows: $$\left[x\right]\coloneqq \{y \in X: y \sim x\}.$$
	
	The element $x$ is called the \emph{equivalence class representative} of $\left[x\right]$.
	
	We define the \emph{quotient set} of $X$ the collection of all the equivalence classes of $X$, and we denote as:
	$$
		\frac{X}{\sim}
		\coloneqq \{\left[x\right]: \, x \in X \}
	.
	$$
\end{defn}

Let's do some considerations from these definition. If an element $y\in X$ is equivalent to $x\in X$, then they are both representative of the same class. Moreover if $y$ belongs to the equivalence class whose $x$ is the representative, again $x$ and $y$ are representative of the same class:
$$ x \sim y \implies \left[x\right] = \left[y\right], \quad y \in \left[x\right] \implies \left[x\right] = \left[y\right].$$

If two equivalence class aren't disjoint, then they are the same equivalence class. If two equivalence class are disjoint, then their representative are not in the equivalence relation:
$$\left[x\right] \cap \left[y\right] \neq \varnothing \implies \left[x\right] = \left[y\right], \quad \left[x\right] \cap \left[y\right] = \varnothing \implies x \nsim y$$

Another trivial result is that the union of all the equivalence class is the set $X$, and the elements of the quotient set are a partition of $X$.

Recalling the previous example:

\begin{exam}
	Considering $\left[x\right]$ as the set of all straight lines parallel to $x$, the quotient set $\frac{X}{\sim}$ can be identified as the set of all the possible direction of the plane.
\end{exam}		
\begin{exam}
	 Considering $\left[x\right]$ as the congruence class of $x$ modulo $q$, fixing for instance $q=5$ we have:
	 $$
	 	\ZZ_5
	 	=\{[0],[1],[2],[3],[4]\}
	 .
	 $$
\end{exam}		
\begin{exam}
	The set of integer numbers can be defined as follows: 
	$$
		\ZZ
		\coloneqq \frac{\NN \times \NN}{\sim}
	.
	$$
\end{exam}		
\begin{exam}
	Also the correct way to define the rational number set is through a quotient set: 
	$$
		\QQ 
		\coloneqq \frac{\ZZ \times \ZZ_0}{\sim}
	;
	$$ 
	indeed, there exist several representations of the same object: $\frac{p'}{q'} = \left[(p,q)\right]$.
\end{exam}


