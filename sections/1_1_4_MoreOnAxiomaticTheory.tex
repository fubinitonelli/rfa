%!TEX root = ../main.tex
\subsubsection{Main results of the axiomatic theory} In our introduction we gave a taste of how deep is the theory behind set theory. Now, as this theory play a role in the fundamentals of analysis we now shows some details of those axioms which allow us to reach useful results. 

Here we will present the aforementioned axiom of choice.

\begin{defn}[Axiom of choice - AC] \label{axiom-of-choice}
	There are different but equivalent formulation for the axiom:
	\begin{itemize}
		\item For any $X \neq \varnothing$ there exists a choice function $f: \Pc(X) \to X$, $f:A \mapsto f(A)\in A$ for any $A \neq \varnothing$;
		\item Let $(E_{\alpha})_{\alpha \in A}$ be a family of non-empty sets $E_\alpha$, indexed by an index set $A$. then we can find a family $(x_\alpha)_{\alpha \in A}$ of elements $x_\alpha$ of $E_\alpha$, indexed by the same set $A$; \footnotemark{}
		\item Any cartesian product of a family on non-empty sets is non-empty: let $\chi = \{X_\alpha\}_{\alpha \in J}$ be a family of sets indexed by a set $J \neq \varnothing$. The cartesian product $\prod_{\alpha \in J} X_\alpha$ is the defined as the set of all mappings $x:J \to \cup_{\alpha \in I} X_\alpha$ such that $x(\alpha) \in X_\alpha$ for each $\alpha \in J$. Each mapping $x$ is defined as choice mapping and $x(\alpha)$ is defined as $\alpha$th coordinate of $x$. Then if $X_\alpha \neq \varnothing$ for each $\alpha \in J$ then $\prod_{\alpha \in J} X_\alpha \neq \varnothing$;
		\item Any member of a family of non-empty sets has at least one element.
	\end{itemize}
\end{defn}
\footnotetext{See: T. Tao, An introduction to Measure Theory, page XV, notation, axiom 0.0.4 .}

This axiom means that it is always possible to choose one element in each set of a collection of sets through a so-called choice function.

The family is assumed to be non-countable; if it is just countable, a weaker version of the axiom, known as the \textit{countable} axiom of choice (CAC), holds.

One interesting consequence arising from the axiom of choice is the following.
\begin{prop}[Banach--Tarski paradox]
	Given a solid ball in 3-dimensional space, there exists a decomposition of the ball into a finite number of disjoint subsets, which can then be put back together in a different way to yield two identical copies of the original ball.
\end{prop}

Despite the existence of this paradox, the axiom of choice is essential to prove even some of the most basic calculus theorems.

\paragraph{Equivalent axioms} Other axioms can be take instead axiom of choice, and some of those has been proved to be equivalent: here we present two of those after having introduced some definitions.

\begin{defn}\label{chain-defn}
	Let $X, \preceq$ be a partially ordered set. His subset $A \subset X$ is called \emph{chain} if it's a totally ordered set.
	
	A chain is \emph{maximal} if it's not properly contained in another chain.
	
	An element $b$ of a partially ordered set $(X, \preceq)$ is an \emph{upper bound} for $A \subset X$ if $a\preceq b$ for all $a \in A$.
	
	A partially ordered set $(X, \preceq)$ is an \emph{inductive set} if every chain has an upper bound.
\end{defn}

\begin{prop} [Hausdorff's maximal principle]
	Every chain in a partially ordered set is contained in a maximal chain.
\end{prop}

\begin{prop} [Zorn's lemma] \label{lemma-zorn}
	Any inductive set has a maximal element.
\end{prop}


\paragraph{The Zermelo--Fraenkel theory and the axiom of choice}  Ernst Zermelo demonstrated in 1904 that every set can be well-ordered if we assume the axiom of choice. In fact, AC is independent from ZF axioms:  Kurt Gödel proved that the negation of AC isn't deducible form ZF in 1938 while Paul Cohen, in 1963, proved that AC neither can be deduced from ZF.

From now on two (different) theories are possibles: the Zermelo--Fraenkel theory with AC (ZFC), in which cardinal numbers has a total order and it is possible to prove a number of important results even if there exist some counter-intuitive results, and the Zermelo--Fraenkel theory without AC. In this last case one can include only the countable axiom of choice and lots of results can be proved anyway.

\paragraph{The continuum hypothesis} Remembering the ordinal number theory, we can assign to who particular transfinite ordinal two specific symbols. 
The least transfinite ordinal is $\omega$ and we denote its cardinality with $\aleph_0$.
The first transfinite uncountable ordinal, that is the set of all countable ordinals, has its cardinality denoted by $\aleph_1$. Those set are themselves ordinals, in the sense that they are representatives from their equivalence classes.

\begin{defn} [Continuum hypothesis (CH)]
	The continuum has the same cardinality of the first transfinite uncountable ordinal, namely:
	$$ 2^{\aleph_0} = \aleph_1 $$
\end{defn}

So, this states that there is not an ``intermediate'' magnitude of infinities between the cardinality of $\NN$ and the cardinality of $\RR$ (see definition \vref{continuum}). 

As for the axiom of choice, in 1940 Kurt Gödel proved that the negation of the CH is not a theorem in ZFC while in 1963 Paul Cohen showed that CF is not a theorem in ZFC; so the CH is another optional axiom of the ZF.

From now on, we will use ZF + AC + CH as our reference theory.
