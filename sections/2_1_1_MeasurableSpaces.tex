%!TEX root = ../main.tex
Here we try to find a way to measure sets. Considering for instance a non-empty set $\Omega$, how can we define a set function $\mu : \Pc(\Omega) \to [0, +\infty]$ which is interpretable as a \emph{measure} of the set $\Omega$? Are there some properties required?

Two reasonable properties for such function are: 
$$
	\mu(\varnothing) 
	= 0, 
	\quad \mu(A \cup B) 
	= \mu(A) + \mu(B)
	\quad \text{ if } 
	\quad A \cap B 
	= \varnothing
.
$$
Does a function with those properties exist?

\subsubsection{Measurable spaces}
Our first step is to define a good environment in which we can have all the tools we need: let's define on which kind of sets we can search for such measure.

\begin{defn}
  Consider a non-empty set $\Omega$.
  The family of subsets $\mm \subseteq \Pc(\Omega)$ is a \emph{$\sigma$-algebra} of $\Omega$ if:
  \begin{itemize}
    \item the \emph{empty set is included}:
    	$$
    		\varnothing \in \mm
    	;
    	$$
    \item it is \emph{closed with respect to complements}:
    	$$
    		E \in \mm 
    		\implies E\comp \in \mm
    	;
    	$$
    \item it is \emph{closed with respect to countable unions}:
     	$$
     		\{ E_i \}_{i \in \NN} \subseteq \mm 
     		\implies \bigcup_{i \in \NN} \, E_i \in \mm
     	.
     	$$
  \end{itemize}
\end{defn}
It can be easily shown from the definition, that also $\Omega$ itself belongs to the $\sigma$-algebra and, by using the De Morgan's laws, that $\mm$ is closed with respect to countable intersections: 
$$
	\forall \{E_j\}_{j\in \NN} \in \mm 
	\qquad \bigcap_{j \in \NN} \, E_j 
	= \bigcap_{j \in \NN} \, (E_j\comp)\comp 
	=\left( \bigcup_{j \in \NN} \, E_j\comp \right)\comp 
	\in \mm
.
$$
Similarly it can be proved that the difference of sets belonging to the $\sigma$-algebra itself belongs to the $\sigma$-algebra.

The power set $\Pc(\Omega)$ is known to be the trivial $\sigma$-algebra.

This allow us to find the ideal environment for a measure:

\begin{defn}
  Let $\Omega \neq \varnothing$ and $\mm \subseteq \Pc(\Omega)$ be a $\sigma$-algebra.\\
  Then $(\Omega, \mm)$ is a \emph{measurable space}.
\end{defn}

It is easy to show that the intersection of $\sigma$-algebras is still a $\sigma$-algebra.

To many purposes is useful to consider the ``smallest'' $\sigma$-algebra containing a given family of subsets of $\Omega$. The following definition formalize this concept.
\begin{defn}
	Let $\Ec \in \Pc(\Omega)$.\\
	The \emph{$\sigma$-algebra generated by $\Ec$}, written as $\sca{\Ec}$, is defined as the intersection of all the $\sigma$-algebras containing $\Ec$.
  \end{defn}
It's easy to prove that this definition is well-posed, to do that, show that the intersection of those $\sigma$-algebras itself is a $\sigma$-algebra. Furthermore, it's easy to show that such $\sigma$-algebra is actually the smallest.

\begin{exer}
  Let $\Omega \neq \varnothing$.\\
  Consider $\Ec = \{ E_i \}_{i=1,\ldots,N}$ as a partition of $\Omega$: prove that $\sca{\Ec}$ has $2^N$ elements.
\end{exer}

Recalling the concepts of open sets and topological spaces (see definition \vref{topological-spaces}), here is presented a special $\sigma$-algebra which has been defined with a special name due to its relevance.
\begin{defn}
  Let $(X, \tau)$ be a topological space with $X \neq \varnothing$. \\
  We call \emph{Borel $\sigma$-algebra}, denoted by $\Bc(X)$, the $\sigma$-algebra generated by open sets.\\
  All the set that it contains are called \emph{Borel sets}.
\end{defn}

Here some general category of sets which belongs to the Borel $\sigma$-algebra. 
\begin{defn}\label{F-sigma-G-delta}
	We call \emph{$F_\sigma$ sets} every countable union of closed sets.\\
	We call \emph{$G_\delta$ sets} every countable intersection of open sets. \footnotemark{}
\end{defn}
\footnotetext{Felix Hausdorff (1868 - 1942) adopted this convection; \textit{F} is for the French word ``fermé'' which means closed, while $\sigma$ is for ``somme'' which means sum (union) in the same language; on the other hand, \textit{G} is for the German word ``Gebiet'' meaning area or neighbourhood, indicating open set, and $\delta$ is for ``Durchschnitt'' which means intersection.}

Open and closed sets, $F_\sigma$ sets and countable intersections of $F_\sigma$ sets, $G_\delta$ sets and countable unions of $G_\delta$ sets all belongs to the Borel $\sigma$-algebra.
