%!TEX root = ../main.tex
\subsubsection{Bases in Hilbert spaces}
In this section we will see that elements of Hilbert spaces can be represented with respect to a series of vectors, known as basis. The idea is the same of Linear Algebra, but now we are dealing with infinite dimensional spaces, thus things require a higher level of technicality and not every operation can be done without some specific assumptions.

\begin{defn}
	Let $(H, \sca{\cdot,\cdot})$ be a Hilbert spaces.\\
	A set of orthogonal vectors $E \subset H$ is \emph{complete} if, given a vector $x \in H$ such that $\sca{x, u} = 0$ for all $u \in E$, then $x = 0$.
	%$$x \in H : \enspace \sca{x, u} = 0 \enspace \forall u \in E \implies x = 0$$
	
	A subset is an \emph{orthogonal basis} if $E$ consists of orthogonal vectors and is complete.\\		
	An orthonormal basis whose vectors have unitary norm (an then consists of orthonormal vectors) is an \emph{orthonormal basis}.
\end{defn}

% If $E$ is complete, then $E^\perp = \{0\}$, hence $\sca{E} ^ \perp= \sca{0}$, and finally $\widebar{\sca{E}}= \{ 0\}$. \\
% Remember that $\sca{E}$ denotes the vector space generated by $E$. \todo{check if this is specified somewhere}

If $E$ is an orthonormal basis of $H$ then $\widebar{\mathrm{span}E}^\perp = E^\perp = \{0\}$. Thus $\widebar{\mathrm{span}E} = H$. \\
One can prove that, if $H$ is separable, a countable orthogonal basis is a Schauder basis.
\begin{theo}
	If a Hilbert space contains at least two distinct vectors, then it has an orthonormal basis.
\end{theo}
We are going to prove the theorem in a constructive way, using Gram--Schmidt orthogonalization, supposing $H$ separable.\\
If $H$ is non-separable,\footnote{``Non-separable Hilbert spaces are quite exotic.'' - M. G.} then the proof requires Zorn's Lemma.
\begin{proof} %not checked
	Let $\{v_n\}_{n \in \NN} \subset H$ be a dense set.\\
	Set 
	$$F_h = \sca{v_1, \ldots, v_h} \quad \forall h \in \NN_0.$$
	Observe that for all $h \in \NN_0$, $F_h$ is a closed subspace of $H$ and $F_h \subset F_{h-1}$.
	
	Take $v_1 \in V_1$ and set 
	$$e_1 = \frac{v_1}{\norm{v_1}}.$$
	In $V_2 \supsetneq V_1$, thanks to the projection theorem, we can find $e_2 \in V_1^\perp$ such that $\norm{e_2} = 1$.\\
	Therefore $\{e_1, e_2\}$ is an orthonormal basis in $V_2$.\\
	Then we proceed by induction, getting an orthonormal basis of $H$.
\end{proof}

\begin{theo}
	All the orthonormal basis in a Hilbert space  have the same cardinality.
\end{theo}
\begin{defn}
	The cardinality of an orthonormal basis of an Hilbert space $(H, \sca{\cdot,\cdot})$ is called \emph{orthogonal dimension} of $H$, and is denoted by $\dim_\perp H$.
\end{defn}
If $H$ is also separable, then $\dim_\perp H \leq \aleph_0$. Indeed, if $u, v \in D$, $u \neq v$, then $\norm{u-v}^2 = \norm{u}^2 +\norm{v}^2 = 2 > 0$. So if $\dim_\perp H \leq \aleph_0$ we cannot find a countable dense set. 


\begin{exam}
	Let $H=l^2$, with $\sca{x, y} = \sum_{i \in \NN} x_i y_i$.\\
	Define the set of sequences $\{\vec e^j\}_{j \in \NN}$ as follows:	
	$$
	e_i^j
	=\begin{cases}
	1 
	& \text{if } i=j \\ 
	0
	& \text{if } i \neq j.
	\end{cases}
	$$
	Then $\{\vec e_j\}_{j \in \NN}$ is an orthonormal basis of $l^2$ and we have:
	$$
	x
	=\sum_{j \in \NN} x_j \vec e^j
	\quad \forall x = \{x_n\}_{n \in \NN} \in l^2.$$
\end{exam}

\begin{exam}
	If $H = L^2((-\pi, \pi))$
	then we can prove that
	$$
		\left\{
			\frac{1}{\sqrt{2\pi}}, 
			\frac{\sin(nt)}{\sqrt \pi}, 
			\frac{\cos(nt)}{\sqrt \pi}
		\right\}_{n \in \NN_0}
	$$
	is an orthonormal basis in $H$. 
\end{exam}

\paragraph{Fourier basis} Now we will use the tools we developed to introduce the Fourier series, which allows to represent continuous function in a different way. In particular, Fourier series is a very powerful tool in engineering since it helps breaking down almost any kind of function in a sum of smooth pieces, which are much easier to treat. This is also very useful when solving partial differential equations for the same reason.

\begin{theo}[Bessel's inequality]
	Let $\{u_n\}_{n \in \NN}$ be a set of orthonormal vectors in $H$.\\
	Then, for all $x \in H$, we have
	$$\sum_{n\in\NN} |\sca{x, u_n}|^2 \leq \norm{x}^2$$
\end{theo}

\begin{proof}
	The absolute value is only really needed when the field is $\CC$, while in this treaty we only used $\RR$. We will however provide a complete proof for the sake of understanding where each term comes from. The reader may revise some concepts, like conjugate in complex numbers, before reading the proof.
	\begin{align*}
	0 & \leq \norm{x-\sum\limits_{n=0}^k \sca{x,u_n} u_n}^2\\
	 & =\sca{x-\sum\limits_{n=0}^k \sca{x,u_n} u_n ,x-\sum\limits_{n=0}^k \sca{x,u_n} u_n} \\
	 & =\sca{x,x} -2\underbrace{\sca{x,\sum\limits_{n=0}^k \sca{x,u_n} u_n}}_{( *)} +\underbrace{\sca{ \sum\limits_{n=0}^k \sca{x,u_n} u_n ,\sum\limits_{n=0}^k \sca{x,u_n} u_n}}_{( **)} =( ***)
	\end{align*}
	The second term becomes
	\begin{align*}
	( *) & =\sum\limits_{n=0}^k \sca{x,\sca{x,u_n} u_n}  & \text{(sesquilinearity)}\\
	 & =\sum\limits_{n=0}^k\overline{\sca{ \sca{x,u_n} x,u_n}} =\sum\limits_{n=0}^k\overline{\sca{x,u_n} \sca{x,u_n}} & \text{(conjugate symmetry)}\\
	 & =\sum\limits_{n=0}^k \sca{ u_n ,x} \overline{\sca{x,u_n}} & \text{(conjugate)}\\
	 & =\sum\limits_{n=0}^k| \sca{x,u_n} |^2 & \text{(conjugate product)}
	\end{align*}
	While the third
	\begin{align*}
	( **) & =\norm{ \sum\limits_{n=0}^k \sca{x,u_n} u_n}^2 \leq \sum\limits_{n=0}^k| \sca{x,u_n} |^2\underbrace{\norm{u_n}^2}_{=1}
	\end{align*}
	Thus
	$$
		(***) \leq \norm{x}^2 -\sum\limits_{n=0}^k| \sca{x,u_n} |^2
	$$
	Letting $k$ to infinity we get the thesis.
\end{proof}


\begin{defn}
	Let $(H, \sca{\cdot, \cdot} )$ be a separable Hilbert space and $\{u_n\}_{n \in \NN}$ one of its orthonormal basis.\\
	We define the \emph{Fourier coefficients} of $x \in H$ with respect to $\{u_n\}_{n \in \NN}$ as follows:
	$$x_n \coloneqq \sca{x, u_n}.$$
\end{defn}

\begin{theo}
	Let $(H, \sca{\cdot, \cdot} )$ be a separable Hilbert space and $\{u_n\}_{n \in \NN}$ one of its orthonormal basis.\\
	Then we can express any $x \in H$ with the following series:
	$$x=\sum_{n\in\NN} \sca{x, v_n} v_n.$$
	The series is known as \emph{Fourier series}.
\end{theo}
Observe that we can also write:
$$
	x = 
	\sum_{n \in \NN} x_n u_n
	\quad \forall x \in H
	\quad \text{ and } \quad
	\sca{x,y} =
	\sum_{n \in \NN} x_n y_n
	\quad \forall x,y \in H
	.
$$
\begin{proof}
	Observe that thanks to Bessel's inequality, $\sum_{n \in \NN} | \sca{x, v_n} |^2 = \sum_{n \in \NN} \abs{x_n}^2$ converges.\\
	Then the sequence of partial sums $S_n = \sum_{j=0}^n x_j vj$ is fundamental:
	$$
	S_n - S_m 
	= \norm{\sum_{j=m+1}^{n}\sca{x, v_j} v_j}^2 
	\leq \sum_{j=m+1}^n |\sca{x, v_j}|^2
	\to 0
	\text{ as } n \to \infty
	.
	$$
	
	Hence $\sum_{n \in \NN} x_n$ converges in $H$ to some $\tilde x$.\\
	Fix $m \in \NN$ for any $n > m$ we have:
	$$
	\sca{x - \sum_{j=0}^{n} x_jv_j, v_m} 
	= x_m - \sca{\sum_{j=0}^n x_j v_j, v_m}
	= x_m - x_m 
	= 0 
	.
	$$ 
%	\begin{align*}
%	\sca{x - \sum_{j=1}^N \sca{x, v_j} v_j, v_{n_0}}
%	&= \sca{x, v_{n_0}} 
%	- \sca{\sca{x, v_{n_0}} v_{n_0}, v_{n_0}} \\
%	&= \sca{x, v_{n_0}} 
%	- (\sca{x, v_{n_0}}) \sca{v_{n_0}, v_{n_0}}
%	= 0
%	\end{align*}
	Taking $n \to +\infty$, we have $\sca{x- \tilde x, v_{m}} = 0$ for all $m \in \NN$, and finally $x= \tilde x$ because $\{u_n\}_{n \in \NN}$ is complete.
\end{proof}

\begin{theo}[Parseval identity]
	Let $(H, \sca{\cdot, \cdot} )$ be a separable Hilbert space and $\{u_n\}_{n \in \NN}$ one of its orthonormal basis.\\
	Then we can write the inner product as the series of the products of the Fourier coefficients, namely:
	$$
	\sca{x, y} 
	= \sum_{n \in \NN}\sca{x, v_n} \sca{y, v_n}
	\quad \forall x, y \in H.
	$$
	This is known as \emph{Parseval identity}.
\end{theo}
\begin{proof}
 	Thanks to orthonormality, for any $n \in \NN$ fixed, we have:
	$$
	\sca{\sum_{j=1}^n x_j v_i, \sum_{j=1}^n x_j v_i}
	= \sum_{j=0}^n \sca{x, v_j} \sca{y, v_j}
	$$
	Letting $n \to +\infty$, the first term converges to $\sca{x, y}$ since the scalar product is continuous in $H \times H$.\\
	The second term is also an inner product in $l^2$, being continuous converges to $\sum_{n\in\NN} x_n y_n$.\\
	We got the thesis.
\end{proof}

From the previous two results we have the following:
\begin{coro}
	Let $(H, \sca{\cdot, \cdot} )$ be a separable Hilbert space.\\
	If $\{u_n\}_{n \in \NN}$ is an orthonormal basis on $H$ then:
	$$
	u_n 
	\wto 0 
	\text{ as }n \to \infty
	.
	$$
	However $\{u_n\}_{n \in \NN}$ does not converges strongly to $0$.
\end{coro}
\begin{proof}
	For any $x \in H$, by Parseval identity we obtain:
	$$\norm{x}^2 
	= \sum_{n \in \NN} x^2_n
	$$
	then $x_n \to 0$ which is $\sca{x, u_n} \to 0$ for each $x \in H$ as $n \to \infty$.
	Thus $u_n \wto 0$.\\
	As $\norm{u_n} = 1$, the sequence cannot converge strongly to $0$.
\end{proof}
\begin{exam}
	We see that $H = L^2((-\pi, \pi))$ has the following orthonormal basis:
	$$
	\left\{
	\frac{1}{\sqrt{2\pi}}, 
	\frac{\sin(nt)}{\sqrt \pi}, 
	\frac{\cos(nt)}{\sqrt \pi}
	\right\}_{n \in \NN_0}
	.
	$$
	Therefore any function $f \in L^2((-\pi, \pi))$ can be written as a sum of its Fourier series, namely
	$$
	f(t)
	= \frac{a_0}{\sqrt{2\pi}}
	+ \sum_{n=1}^\infty \frac{a_n}{\sqrt{\pi}} \cos(nt)
	+ \sum_{n=1}^\infty \frac{b_n}{\sqrt{\pi}} \sin(nt)
	$$
	and the convergence is in $L^2((-\pi,\pi))$.
	
	The Fourier coefficient are given by:
	\begin{gather*}
	a_0
	= \frac{1}{\sqrt{2\pi}}\int_{-\pi}^\pi f(t)\, \dt
	, \\
	a_n
	= \frac 1 \pi \int_{-\pi}^\pi f(t) \cos(nt) \, \dt
	, \quad
	b_n
	= \frac 1 \pi \int_{-\pi}^\pi f(t) \sin(nt) \, \dt
	.
	\end{gather*}
	
	One can prove that if $f \in \Cc^2([-\pi, \pi])$ then its Fourier series converges to $f$ uniformly.\\
	Just integrate by part the integrals defining $a_n$ and $b_n$ twice.
\end{exam}

\paragraph{Finite dimensional projectors}
Here we will see how we can also project elements onto finte dimensional spaces.
This will be useful to make many useful counterexamples and understand some concepts.\\
Let $(H, \sca{ \cdot, \cdot})$ be a Hilbert space, and $\{v_n\}_{n \in \NN}$ be an orthonormal basis.

Set
$$
	H_N
	= \Span\{v_1,\ldots, v_N \}
.$$
This is a closed subspace: an explicit representation of the projection on its is:
$$
P_{H_N} x 
= \sum_{n=1}^N \sca{x, v_n} v_n
= \sum_{n=1}^N x_n v_n
.
$$
Notice that $P_{H_N} \in \Kc(H)$ since it is finite-rank, and 
$$
\norm{x-P_{H_N} x}
\to 0
$$
as $N \to +\infty$, but $\{P_N\}_{N \in \NN_0}$ does not converge in $\Bc(H)$ if $\dim_\perp H = \aleph_0$.
Indeed it would converge to the identity operator which is not compact while the projectors $P_{H_N}$ are.


\paragraph{The space $l^2$ is a model of a separable Hilbert space} \todo{bla bla bla}\\
Remember that any separable Hilbert space has a Schauder basis.

\begin{prop}
	Let $(H, \sca{\cdot , \cdot})$ be a separable Hilbert space. \\
	Then it is linearly isomorphic and isometric to $l^2$.
\end{prop}

\begin{proof}
	Let $\{v_n\}_{n \in \NN} \subset H$ be an orthonormal basis of $H$.\\
	The mapping $\Fc : H \to l^2$, defined by
	$$
	x 
	\mapsto \Fc x 
	= \{x_n\}_{n \in \NN}$$
	where $x_n$ is the $n$-th Fourier coefficient of $x$, is well defined, due to Bessel's inequality.\\
	Notice that it is also injective and surjective.
	
	Let $\{a_n\}_{n \in \NN} \in l^2$ and consider
	$$
	\sum_{n\in\NN}a_n u_n
	$$
	where $\{u_n\}_{n \in \NN}$ is an orthonormal basis in $H$.
	
	The series converges to some $x \in H$, indeed:
	$$
	\norm{\sum_{n\in \NN} a_n u_n}^2 \leq \sum_{n\in \NN}|a_n|^2 \norm{u_n}^2 = \sum_{n\in \NN}|a_n|^2 = \norm{\{a_n\}}_{l^2}^2 < +\infty 
	$$
	
	Moreover, for any fixed $m \in \NN$ we have:
	$$
		\sca{\sum_{k=0}^n a_k u_k, u_m}
		= a_m
	$$
	for $n > m$.
	
	Letting $n$ go to $\infty$ we find $x_m = \sca{x, u_m} = a_m$.
	
	Thus
	$$
	\Fc x 
	= \{a_n\}_{n \in \NN}.
	$$
	Finally we also use the bounded inverse mapping theorem (see \vref{bounded-inverse-theo}) to prove that the inverse is also bounded.
	
%	We need to prove that $\Fc$ is linear.
%	
%	$\Fc$ is well defined, due to Bessel's inequality. Moreover, from the Parseval's identity, we have that $\norm{x^2}=\sum_{n \in \NN} \hat x_n^2$. Therefore $\norm{\Fc x}_{l^2} = \norm{x}_H$, and $\Fc$ is an isometry.
%	
%	Indeed, let $\{\tilde x_n\}_{n \in \NN} \subset l^2$ and consider the series $\sum_{n\in\NN} \tilde x_n v_n$, whose character is still unknown. Notice that, taking $M> N$, the sequence of partial sums is Cauchy:
%	$$\norm{\sum_{n=0}^M \tilde x_n v_n - \sum_{n=0}^N \tilde x_n v_n}^2
%	= \norm{\sum_{n=N+1}^M \tilde x_n v_n}^2
%	\overset{(Py)}{=} \sum_{N=N+1}^M x_n^2 \xrightarrow{N,M \to +\infty} 0$$
%	This implies $\sum_{n \in \NN} \tilde x_n n_n = \tilde x$. But $\tilde x_n = \sca{\tilde x, v_n}$, hence $\exists \, \tilde x = \Fc^{-1}(\{\tilde x_n \}_{n \in \NN})$, and $\Fc$ is surjective.
\end{proof}


%\begin{rema}
%	Let $(H, \sca{ \cdot, \cdot,})$ be a Hilbert space, and $\{v_n\}_{n \in \NN}$ be an orthonormal basis. \\
%	Then $\hat x_n = \sca{x,v_n} \to 0 \enspace \forall x \in H$ as $n \to +\infty$, because $\sum_{n \in \NN} | \hat x_n |^2 < +\infty$ . Via Riesz, that is equivalent to $v_n \wto 0$, but of course $v_n \not \to 0$, as $\norm{v_n} = 1$.
%	
%	In particular $\sin(nx) \wto 0$ and $\cos(nx) \wto 0$.
%\end{rema}
%
%\begin{prop} 
%	A compact set $K$ in a infinite-dimensional Banach space is always nowhere dense set, \textit{i.e.} $\mathring K = \varnothing$.
%\end{prop}
%
%\begin{rema}
%	Any separable Hilbert space has a Schauder's basis: this can be easily deduced from the isometric isomorphism with $l^2$.
%\end{rema}



