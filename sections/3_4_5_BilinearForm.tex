%!TEX root = ../main.tex
\subsubsection{Bilinear Forms and Lax Milgram lemma}
Here we will generalize the concept of inner product. In fact, the following definition will make it clear that the inner product is an operation which belongs to a much general class.

\begin{defn}
	Let $(H, \sca{\cdot, \cdot})$ be a Hilbert space.\\
	An application $a: \, H \times H \to \RR$ is called \emph{bilinear form} if:
	\begin{align*}
		u &\mapsto a(u,v) \text{ is linear for each fixed } v \in H,\\
		v &\mapsto a(u,v) \text{ is linear for each fixed } u \in H.
	\end{align*}
\end{defn}

\begin{defn}
	A bilinear form $a: \, H \times H \to \RR$ is \emph{coercive} if there exists $\alpha > 0$ such that:
	$$a(u,u) \geq \alpha\norm{u}^2 \quad \forall u \in H.$$
\end{defn}

\begin{prop}
	A bilinear form $a: \, H \times H \to \RR$ is continuous if and only if there exists $M > 0$ such that:
	$$ \abs{a(u,v)} \leq M\norm{u}\norm{v} \quad \forall u, v \in H.$$
\end{prop}

Observe that the inner product of an Hilbert space is a continuous and coercive bilinear form.

Consider $H = \RR^N$ with the euclidean scalar product $\sca{\cdot, \cdot}_2$ and a matrix $A \in \RR^{N \times N}$.
Then $a(x,y) = \sca{Ax, y}_2$ for all $x, y \in \RR^N$ is a continuous bilinear form.

The following is a lemma which generalize the last representation theorem.

\begin{theo}[Lax--Milgram]\label{theo-lax-milgram}
	Let $a:\, H \times H \to \RR$ be a continuous and coercive bilinear form.\\
	Then for any $l \in H^\star$ there exists a unique $u \in H$ such that:
	$$a(u,v) = Lv \quad \forall v \in H.$$
	Moreover we have
	$$\norm{u} \leq \frac{1}{\alpha}\norm{L}_\star$$
	where $\alpha$ is the coercivity constant.
\end{theo}

\begin{proof}
	Fix $u \in H$ and observe that:
	$$
		v \in H:\
		v \mapsto a(u,v) \in \RR
	$$
	is an element of $H^\star$.
	Therefore there is a unique $h \in H$ such that:
	$$
		a(u,v) 
		= \sca{h,v} 
		\quad \forall v \in H
		.
	$$
	Thus we can define a linear operator $A:H \to H$ by setting $Au = h$. This operator is also bounded. Indeed:
	$$
		\norm{Au}^2 
		= \sca{Au, Au} 
		= a(u, Au) 
		\leq M \norm{u} \norm{Au}
	$$
	so that $\norm{Au} \leq M \norm{u}$.
	
	Moreover, we have:
	$$
		\alpha\norm{u}^2 
		\leq a(u,u) 
		= \sca{Au, u} 
		\leq \norm{Au}\norm{u}
	$$
	so $A$ is also injective.
	
	On the other hand, there is a unique $w \in H$ such that (see Riesz's representation theorem \vref{riesz-repr}):
	$$
		Lv 
		= \sca{w, v} 
		\quad \forall v \in V
		.
	$$
	Therefore the thesis can be reformulated as follows: for any $w \in H$ find a unique $u \in H$ such that:
	$$
		\sca{Au, v} 
		= \sca{w,v} 
		\quad \forall v \, \in V
	$$
	so we're left to show that $A$ is surjective.
	
	Let's prove first that $\Im(A)$ is a closed subspace of $H$.\\
	Indeed let $\{y_n\}_{n\in \NN} \subset \Im(A)$ be such that $y_n \to y$ as $n \to \infty$.\\
	We know that there exists a unique $u_n \in H$ such that $A u_n = y_n$.\\
	Moreover, we have
	$$
		\alpha \norm{u_n - u_m} 
		\leq \norm{A u_n - A u_m} 
		= \norm{y_n - y_m}
	$$
	therefore $\{u_n\}_{n \in \NN}$ converges in $H$ to some $u$ and, because $A$ is continuous, it follows
	$$
		A u_n 
		\to A u 
		\quad n \to \infty
		.
	$$
	Hence $y=Au \in \Im(A)$ and $\Im(A)$ is closed.
	
	We can now prove that $A$ is surjective.\\
	By contradiction suppose that $\Im(A) \subsetneq H$. Then there exists $y_0 \in H$ such that $y_0 \neq 0$ and $y_0 \in \Im(A)^\perp$. However, we have
	$$
		0 
		= \sca{A y_0, y_0} = a(y_0,y_0) 
		\geq \alpha\norm{y_0}^2 
		\implies y_0 = 0
		.
	$$
	This is a contradiction.
	
	Finally, we recall that
	$$
		\alpha\norm{u} 
		\leq \norm{Au} 
		= \norm{w} 
		= \norm{L}_\star
		.
	$$
\end{proof}

\paragraph{Inner product} Even if a scalar product is a continuous and coercive bilinear form, a continuous and coercive bilinear form is not necessarily a scalar product: this because it might be not symmetric.\\
Due to this point, the Lax--Milgram lemma can be seen as a generalization of the Riesz representation theorem (\vref{theo-projection}).

\begin{prop}
	Let $a: \, H \times H \to \RR$ be a continuous and coercive bilinear form.\\
	If $a$ is symmetric, namely:
	$$
		a(u,v) 
		= a(v,u) 
		\quad \forall u,v \in H
		.
	$$
	then $a$ is a scalar product in $H$ and its induced norm is equivalent to the norm in $H$.
\end{prop}

\begin{proof}
	By coercivity and the other hypothesis we have:
	$$
		\alpha \norm{u}^2 
		\leq a(u,u) \leq M\norm{u}^2 
		\quad \forall u \in H
	$$
	where $\norm{\cdot}$ is the norm in $H$.\\
	
	Hence $a(u,u) \geq 0$ for all $u \in H$ and $a(u,u) = 0$ if and only if $u = 0$.
	
	Thus $a(\cdot, \cdot)$ is a scalar product in $H$ and its induced norm
	$$
		\norm{u}_a 
		= \sqrt{a(u,u)}
	$$
	is equivalent to $\norm{\cdot}$.
\end{proof}


\paragraph{An application of the Lax--Milgram lemma}\label{app-lax-milgram} All this theory has a practical purpose. Here we will see in action the tools we built solving a differential problem.\\
Consider the following boundary value problem: find $u$ such that
$$
	\begin{cases}
	-(  \alpha(x) u' + \beta(x) u)' + \gamma(x)u 
	= f(x)
	\quad x \in (a,b)\\
	u(a) 
	= u(b) 
	= 0
	\end{cases}
$$
where $\alpha, \beta \in \Cc^1([a,b])$ and $\gamma, f \in C([a,b])$.

Suppose $w \in \Cc^2([a,b])$ is a classical solution to the problem.

Consider $\phi \in \Cc^\infty((a,b))$ multiply the equation by $\phi$ and integrate over $(a,b)$.
Integrating by parts we get the identity:
$$
	\int_a^b \left[
	\left(\alpha(x)w'(x)
	+ \beta(x)w(x) \right)\phi'(x)
	+ \gamma(x)w(x)\phi(x)\right]\,\dx 
	= \int_a^bf(x) \phi(x) \, \dx
	.
$$

Setting
$$
	V_0
	= \{v \in H^1((a,b))
	: \ v(a)
	=v(b)
	=0\}
$$
where $H^1((a,b)) = W^{1,2}((a,b))$, we observe that $V_0$ is a closed subspace of $H^1((a,b))$. Moreover the scalar product
$$
	(v_1, v_2) 
	= \int_a^b v_1'(x) v_2'(x) \, \dx
$$
induces a norm
$$
	\norm{v}_0 
	= \norm{v'}_2
$$
which is equivalent to the standard norm in $H^1((a,b))$.\\
(This because the Poincaré's inequality $\norm{v}_2 \leq (b-a)\norm{v'}_2$.)\\

We can prove that $C_C^\infty((a,b))$ is dense in $V_0$ (this is a density theorem: density theorems allow to establish approximations) which allows us to deduce that our solution $w$ satisfies the identity
$$ 
	\int_a^b [(\alpha(x)w'(x) + \beta(x)w(x))v'(x) + \gamma(x)w(x)v(x)]\,\dx = \int_a^bf(x) v(x) \, \dx \quad \forall v \in V_0. 
$$

We can now look at our problem from a more general standpoint.

Let us set, for all $u,v \in V_0$
$$
	a(u,v)
	= \int_a^b [(\alpha(x)u'(x) 
	+ \beta(x)u(x))v'(x) 
	+ \gamma(x)u(x)v(x)]\,\dx
$$
and observe that $a: \, V_0 \times V_0 \to \RR$ is a bilinear form.

Thus we can give to the original problem a weaker formulation, namely find $w \in V_0$ such that
$$
	a(w,v)
	=Lv
	\quad \forall v \in V_0
$$
where $L \in V^\star_0$ is defined by 
$$
	Lv 
	= \int_a^b f(x)v(x)\, \dx 
	\quad \forall v \in V_0
	.
$$

Now we want to apply Lax--Milgram lemma. Let's check the hypothesis.

Observe that $a(\cdot, \cdot)$ is continuous as 
\begin{align*}
	\abs{a(u,v)} 
	&\leq \int_a^b \abs{(\alpha(x)u'(x) 
	+ \beta(x)u(x))v'(x) 
	+ \gamma(x)u(x)v(x)}\,\dx\\
	&\leq \norm{\alpha}_\infty \int_a^b \abs{u'v'}\,\dx
	+\norm{\beta}_\infty \int_a^b \abs{uv'}\,\dx
	+\norm{\gamma}_\infty \int_a^b \abs{uv}\,\dx\\
	&\leq \norm{\alpha}_\infty \norm{u'}_2 \norm{v'}_2
	+ \norm{\beta}_\infty \norm{u}_2 \norm{v'}_2
	+ \norm{\gamma}_\infty \norm{u}_2 \norm{v}_2
\end{align*}

for all $u,v \in V_0$.\\
Then, using again the Poincaré's inequality we have:
$$
	\abs{a(u,v)}\leq \norm{\alpha}_\infty \norm{u'}_2 \norm{v'}_2
	+ (b-a)\norm{\beta}_\infty \norm{u'}_2 \norm{v'}_2
	+ (b-a)^2 \norm{\gamma}_\infty \norm{u'}_2 \norm{v'}_2
	.
$$

Hence, setting 
$$
	M = \norm{\alpha}_\infty
	+ (b-a)\norm{\beta}_\infty
	+ (b-a)^2\norm{\gamma}_\infty
	,
$$
we have
$$
	\abs{a(u,b)} 
	\leq M \norm{u}_0\norm{v}_0 
	\quad \forall u, v \in V_0
	.
$$

We can add other assumption to the coefficients $\alpha$, $\beta$ and $\gamma$ that hold for any $x \in [a,b]$, like the following:
$$
	\alpha_0
	= \min\limits_{x\in[a,b]} \alpha(x)
	> (b-a) \norm{b}_\infty
	\text{ and }
	\gamma(x) 
	\geq 0
	.
$$

Collecting the results we obtained we have:
\begin{align*}
	a(u,u)
	&= \int_a^b \alpha(x)(u'(x))^2 \, \dx
	+ \int_a^b \beta(x)u(x)u'(x) \, \dx
	+ \int_a^b \gamma(x)(u(x))^2 \, \dx \\
	& \geq \alpha_0\norm{u'}_2
	+ \int_a^b \beta(x)u(x)u'(x) \, \dx.
\end{align*}

We can find an upper bound for the middle term as well:
$$
	\abs{\int_a^b \beta(x)u(x)u'(x) \, \dx}
	\leq \norm{\beta}_\infty \norm{u}_2 \norm{u'}_2 
	\leq (b-a) \norm{\beta}_\infty \norm{u'}_2^2
	,
	$$
	so we get:
	$$
	a(u,u)
	\geq [\alpha_0 -(b-a)\norm{\beta}_\infty] \, \norm{u'}_2
	$$
	we can set:
	$$
	\alpha 
	= [\alpha_0 -(b-a)\norm{\beta}_\infty]
	> 0
	.
$$

Then, finally we have:
$$
	a(u,u)
	\geq \alpha \norm{u}_0
	.
$$
and we can conclude that $a(\cdot, \cdot)$ is coercive.

Now we can apply the Lax--Milgram lemma and we deduce the following proposition:
If $\alpha$, $\beta$, $\gamma$, $f \in C([a,b])$ and 
$$
	\min\limits_{x\in[a,b]} \alpha(x)
	> (b-a) \norm{b}_\infty
	\text{ and }
	\gamma(x) 
	\geq 0
	\quad \forall x \in [a,b]
	$$
	then there exists a unique $u \in V_0$ such that:
	$$
	a(u,v) 
	= Lv \quad \forall v \in V_0
	.
$$

The solution $u$ given by the Lax--Milgram lemma is a weak solution to the original boundary value problem.

We can further weaken the assumptions on the coefficients by taking $\alpha$, $\beta$, $\gamma \in L^\infty((a,b))$.

In addiction $f$ can be taken in $V_o^\star$: assume $f \in L^2((a,b))$ or even a linear bounded functional link, for instance,
$$
	Lv
	= v(x_0)
	\quad \forall v \in V_0
$$
where $x_0 \in (a,b)$. It's easy to check that $L \in V_0^\star$.

Consider now the linear application $K:L^2((a,b)) \to L^2((a,b))$
defined by
$K:f\mapsto u$
where $u$ satisfies
$$
	a(u,v) 
	= \int_a^b fv 
	\quad \forall v \in V_0
	.
$$

We have that $\Im(K) \subseteq V_0$ and we know that 
$$
	V_0
	\hookrightarrow \hookrightarrow C([a,b])
	\hookrightarrow L^2((a,b))
	.
$$

Therefore
$$
	V_0
	\hookrightarrow \hookrightarrow L^2((a,b))
	.
$$
Hence $K$ is a compact operator.

From Lax--Milgram we have also a stability estimate:
$$
	\norm{u}_0
	\leq \frac 1 \alpha \norm{L}_{V_0^\star}
	.
$$

If $f \in L^2((a,b))$ then, using Poincaré's inequality:
$$
	\abs{Lv}
	= \abs{\int_a^b f v \, \dx}
	\leq \norm{f}_2\norm{v}_2
	\leq (b-a)\norm{f}_2\norm{v}_0
$$
so that
$$
	\norm{L}_\star
	\leq (b-a)\norm{f}_2
$$
and
$$
	\norm{u}_0
	\leq \frac{(b-a)}{\alpha} \norm{f}_2
	.
$$

If
$$Lv 
	= v(x_0) 
	\quad \forall v \in V_0
$$

where $x_0 \in (a,b)$, then, using Hölder's inequality:
$$
	\abs{Lv}
	= \abs{v(x_0)}
	= \abs{\int_a^{x_0} v'(y) \, \dy}
	\leq (b-a)^{(\frac 1 2)} \norm{v}_0
$$

so that
$$
	\norm{L}_\star
	\leq (b-a)^{\frac 1 2}
$$
and
$$
	\norm{u}_0
	\leq
	\frac{(b-a)^{\frac 1 2}}{\alpha}
	.
$$

The bilinear form $a(\cdot, \cdot)$ is not symmetric because of $\beta$.\\
If $\beta \equiv 0$ then we deal with following boundary value problem
$$
	\begin{cases}
	-(\alpha(x)u')' 
	+ \gamma(x)u 
	= f(x) 
	\quad x \in (a,b)\\
	u(a) 
	= u(b) 
	= 0
	\end{cases}
$$
and its weak formulation has a unique solution provided that, for instance
$$
	\alpha = \min_{x\in [a,b]} \alpha (x) > 0
	, 
	\quad \gamma(x)\geq 0 
	\quad x \in [a,b]
	.
$$
However, in this case we can just apply the Riesz representation theorem since $a(\cdot, \cdot)$ is an inner product in $V_0$ whose induced norm is equivalent to $\norm{\cdot}_0$.
