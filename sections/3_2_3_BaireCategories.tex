%!TEX root = ../main.tex
\subsubsection{Baire's categories} This section shows some results which are another insight in the algebraic structures we have worked with. In this section $X$ can be either a normed vector space $(X, \norm{\cdot}_X)$ or a metric space $(X,d)$.

First, recall the definition of nowhere dense set (\vref{defn-density}); $E\subset X$ is \textit{nowhere dense} if $\mathring{\bar{E}} = \varnothing$ (the interior of the closure of $E$ is empty), or equivalently if the closure of $E$ does not contain any ball.\\
Observe that this is not true of dense set, consider for example $\QQ$ in $\RR$, the closure of the internal point of $\QQ$ is not empty.
%Observe that for a nowhere dense set the set of internal points of its closure is empty, but in general it is not empty the closure set of its internal point. Consider for example $\QQ$ in $\RR$.
Indeed, for $X=\RR$, $\NN$ and $\ZZ$ are nowhere dense.

For any Banach space $X$, a finite-dimensional subspace $Y \subsetneq X$ is nowhere dense with respect to the topology of $X$.

A compact set in an infinite-dimensional Banach space is nowhere dense also.

Every subspace $V \subsetneq \RR^N$ of dimension $M<N$ is nowhere dense, with respect to the topology of $\RR^N$.

Both the Cantor set and its generalized counterpart are nowhere dense.

\paragraph{First and second category} Consider the following definition.
\begin{defn}
	We say that $E\subset X$ is \emph{first category} if it is a countable union of nowhere dense sets.\\
	Otherwise we say that $E$ is \emph{second category}.
\end{defn}
First category sets are very small in measure: they are also called \emph{meagre sets}.

\begin{exam}
	Consider $X = \RR$ and $E=\QQ$. Notice that $\bar{E}=\bar{\QQ}=\RR$, and $\mathring{\bar{\QQ}}=\RR$, thus $\QQ$ is not nowhere dense: $\QQ$ is first category, since is a countable union of single points.\\
	Notice that $\QQ$ is an $F_\sigma$ set, but it is not closed. \\
	Moreover, $\RR \setminus \QQ = \bigcap_{n\in \NN} A_n$, where $A_n=\{x_n\}\comp$. Every $A_n$ is open and dense, and thus $\RR\setminus \QQ$ is $G_\delta$, but it is not open (see definition \vref{F-sigma-G-delta}).
\end{exam}

\begin{prop}
	There exists subsets $E \subset \RR$ which are either first category and $\lambda(E)>0$ or second category and $\lambda(E)>0$.\\
	Also $\RR$ can be written as follows: $$\RR=E_1 \cup E_2$$ where $E_1$ is first category and $\lambda(E_2)=0$. 
\end{prop}

Observe that the union of two first category is still first category, and $\RR$ is not first category. Thus $E_2$ must be second category.

\paragraph{Baire's theorem} The following is a topological theorem which is a cornerstone in functional analysis.
\begin{theo}[Baire]
	Any complete metric space is second category with respect to its metric.
\end{theo}

\begin{proof}
	Let $X$ be the space in question. By contradiction, suppose:
	$$X=\bigcup_{n=1}^{\infty} C_n \quad \text{with } C_n \text{ closed and such that } \mathring{C_n}=\varnothing \quad \forall n \in \NN_0.$$
	Then define $A_n = C_n\comp$ which are open $\forall n \in \NN_0$.
	
	Let $x_1\in A_1$. Then there exists $\varepsilon_1 < 1$ such that $\overline{B(x_1, \varepsilon_1)} \subset A_1$.\\
	Since $C_2$ cannot contain all $B(x_1,\varepsilon_1)$, then there exists $x_2$ contained in the open set $A_2 \cap B(x_1,\varepsilon_1) \neq \varnothing$.\\
	Thus we can find $\varepsilon_2 < \frac 1 2$ such that $\overline{ B(x_2,\varepsilon_2)}\subset A_2 \cap B(x_1,\varepsilon_1)$.\\
	Iterating this argument, we can construct a decreasing sequence of closed balls:
	$$\overline{B(x_1,\varepsilon_1)} \supset \overline{B(x_2,\varepsilon_2)} \supset \overline{B(x_3,\varepsilon_3)} \supset  \cdots  \supset \overline{B(x_n,\varepsilon_n)} \supset  \cdots$$
	such that $\varepsilon_n < \frac{1}{2^n}$ and $\overline{B(x_{n+1},\varepsilon_{n+1})} \subset A_{n+1} \cap B(x_n, \varepsilon_n)$.
	
	Therefore $\{x_n\}_{x\in\NN_0}$ is a Cauchy sequence as $$d(x_n, x_m) \leq \frac{1}{2^{m-1}} \quad n \geq m \geq 1,$$
	so $x_n \to x$ for some $x\in X$, because $X$ is complete. Thus there exists $n_0 \in \NN_0$ such that $x\in C_{n_0}$.
	
	On the other hand,  we have $x_n \subset \overline{B(x_{n_0}, \varepsilon_{n_0})}\subset A_{n_0} =C_{n_0}\comp \enspace \forall n \geq n_0 $, and thus $\{x\}=\cap_{n\geq n_0} B(x_n, \varepsilon_n) \subset A$, so $x$ also belongs to $A_{n_0}= C_{n_0}\comp$, which is a contradiction.
\end{proof}

\begin{coro}
	An infinite dimensional Banach space cannot have a countable algebraic basis.
\end{coro}
\begin{proof}
	Suppose that exists a sequence $\{x_n\}_{n\in\NN}\subset X$ is a Hamel basis of the Banach space $X$ and consider the space generated by the linear combination of those vectors, namely:
	$$V_n = \sca{\{x_0,x_1,\ldots,x_n,\ldots\}}$$
	Then $\mathring{\widebar{V}}_n=\mathring{V}_n=\varnothing$ for any $n$, but $X=\bigcup_{n\in\NN}V_n$, so it is first category. This contradicts Baire's theorem, and thus the supposition is false.
\end{proof}

\begin{coro} \label{coro-inters-dense}
	Let $(X,\norm{\cdot})$ be a Banach space (or $(X,d)$ a complete metric space).\\
	Then, the intersection of any countable family of open dense sets is dense.
\end{coro}
\begin{proof}
	Let $\{A_n\}_{n \in \NN}$ such that $A_n$ open and dense $({\bar A}_n=X)$ $\forall n \in \NN$. Suppose by contradiction that:
	$$C=\overline{\bigcap_{n\in\NN}A_n} \subsetneq X$$
	Thus $C$ is closed, $\mathring{C}$ is open, and there exists $B \subset E\comp$ closed ball in $X$ such that:
	$$\left(\bigcap_{n\in \NN}A_n\right) \cap B = \varnothing.$$
	Then:
	$$
	\bigcup_{n \in \NN}\left(A_n\comp \cap B\right)= B.
	$$
	We also have:
	$$\mathring{\overline{A_n\comp \cap B}} \subset \mathring{\overline{A_n\comp}}= \mathring{A_n\comp} = \varnothing$$
	and so
	$$
	\mathring{\overline{A_n\comp \cap B}} = \mathring{A_n\comp}.$$
	If $\mathring{A_n\comp}$ wouldn't be the emptyset, there would exist a ball $\tilde B$ such that $\tilde B \subset A_n\comp$, but $\widebar{A_n}=X$.
	Therefore, $\mathring{\overline{A_n\comp \cap B}} = \varnothing$, $(A_n\comp \cap B)$ is nowhere dense, and we deduce that $B$ is first category.
	However, $B$ is complete metric space since it's closed, and because of Baire's theorem it should be second category, which is a contradiction.
\end{proof}

As an exercise, consider $X$ Banach space and $C\subset X$ closed. Prove that $\mathring C = \varnothing$ if and only if $A=C\comp$ is dense.
