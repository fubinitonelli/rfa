%!TEX root = ../main.tex
\subsubsection{Topological spaces}

Here we try a different approach to deal with ``closedness'': alongside metric spaces we define the topological spaces from which we will obtain similar results.

\begin{defn} \label{topological-spaces}
  Let $x \neq \varnothing$ and $\tau \subseteq \Pc(X)$.\\
  We say that $\tau$ is a \emph{topology} if:
  \begin{itemize}
    \item it contains the set itself and the empty set: 			
    $$
    	X \in \tau,
    	\quad \varnothing \in \tau
    ;
    $$
    \item it is closed with respect to finite intersections: 	
    $$
    	A, B \in \tau 
    	\implies A \cap B \in \tau
    ;
    $$
    \item it is closed with respect to uncountable unions: 		
    $$
    	\{ A_i \}_{i \in I} \subset \tau 
    	\implies \bigcup_{i \in I} \, A_i \in \tau
    .
    $$
  \end{itemize}
  In this case, we also say that $(X, \tau)$ is a \emph{topological space}.\\
  Moreover, we call \emph{open sets} each set belonging to $\tau$.
\end{defn}

As we see, this definition also contains the notion of openness, and here is generalized (see definition \vref{open-close-set-metric-spaces} for the definition of open set in metric spaces). Anyway, there exists a link between metric spaces and topological spaces:
\begin{defn}\label{rem-metric-spaces-topology}
	Let $(X,d)$ be a metric space.\\
	We define the \emph{topology induced by the distance} $d$ as follows:
	$$
	\tau
	= \{ A \subset X : A \text{ is open with respect to } d\}
	.
	$$
\end{defn}

We can prove that two metric spaces $(X, d_1)$ and $\left(X, d_2\right)$, with $d_1$ equivalent to $d_2$ (see definition \vref{equivalent-metrics}), produce the same topology; this result is obtainable using the characterization of the closure in metric spaces.

As we define the open set as the set belonging to the topology, closed sets are the sets whose complement are open, namely $A \subset X$ is closed if $X \setminus A$ is open ($X \setminus A \in \tau$).

We can also identify a topological space by considering its closed subsets instead of the open ones. This is possible by the De Morgan's law (see section \vref{de-morgans-laws}) which allow us to define the \textit{dual properties}. Consider $\Cc$ as the collection of all the closed subsets of $X$, then if satisfy the dual properties, which are: $$X, \varnothing, X \in \Cc; \quad C, D \in \Cc \implies C \cup D \in \Cc; \quad \{ D_i \} \text{ family of }\Cc \implies \cap_i D_i \in \Cc;$$
then the family of the complements $\{ f\comp : f \in \Cc \}$ is a topology.

Observe that $\varnothing$ and $X$ are \textit{both} open and closed.

Consider now some examples: if $X \neq \varnothing$, $\tau = \Pc(X)$ is the topology associated with the discrete metric (indeed, all the set are open with respect to this metric), while $\tau = \{ \varnothing, X \}$ is called the \emph{trivial topology}; let $X = \{ 1, \ldots, 4 \}$: a possible topology of $X$ is given by $\tau = \{ \varnothing, X, \{ 1 \}, \{ 2, 3 \}, \{ 1, 2, 3\} \}$.

% At last, a topology can be induced by a generic set:
% \begin{defn}
% 	Let $(X,\tau)$ be a topological space.\\
% 	We define the \emph{topology induced by the a set} $E$ as follow:
% 	$$
% 	\tau_E 
% 	= \{ A \subset X : A = U \cap E \text{ with } U \in \tau\}
% 	.
% 	$$
% \end{defn} \todo{check}
