%!TEX root = ../main.tex
\subsubsection{Derivative of a measure: properties}
Now we are ready to proceed in theory development about derivatives. With the definition of almost everywhere we can talk about uniqueness and some properties.

\begin{prop}
	If the hypothesis of the Radon--Nikodym theorem (\vref{theo-radon-nikodym}) hold, then the Radon--Nikodym derivative is unique almost everywhere.
\end{prop}
The reader can easily prove this result using proposition \vref{integral-inequality-sigma-finite}. Do it now before see the following proof!

\begin{proof}
	Consider two Radon--Nikodym derivatives for $\frac{\dnu}{\dmu}$: two positive and measurable function $\phi_1$ and $\phi_2$.\\
	Then we have: $$\int_E \phi_1 \dmu = \int_E \phi_2 \dmu \quad \text{for all } E \in \mm.$$
	From the previously referenced proposition we have both $\phi_1 \geq \phi_2$ and $\phi_2 \geq \phi_1$ a.e.\ in $\Omega$, hence the thesis.
\end{proof}

Moreover, it holds the following.
\begin{prop}
	If the Radon--Nikodym derivative $\phi$ exists and $\mu (\Omega) < +\infty$, then $\phi \in L^1(\Omega, \mm, \mu)$. 
\end{prop}
%
%\begin{rema}
%	$f:\Omega \to \RR$ is measurable. Then:
%	$$f \in L^1(\Omega, \mm, \nu) \iff f \frac{\de\nu}{\de\mu} \in L^1(\Omega, \mm, \mu)$$
%\end{rema}
%The proof of this result is also left as an exercise for the reader.

\paragraph{Basic properties} Here we present some properties of the Radon--Nikodym derivative.

\begin{prop}[Change of measure]
	For any measurable positive function $f:\Omega \to [0,\infty]$ we have:
	$$ \int_\Omega f \, \de \nu = \int_\Omega f \, \frac{\de \nu}{\de \mu} \de \mu.$$
\end{prop}

\begin{prop}[Linearity of the derivative]
	For all $c_1, c_2 \geq 0$ we have:
	$$\dfrac{\de(c_1\nu_1+c_2\nu_2)}{\de\nu} = c_1 \dfrac{\de\nu_1}{\de\mu}+ c_2 \dfrac{\de\nu_2}{\de\mu}.$$
\end{prop}

\begin{prop}[Chain rule]
	Consider three measures $\lambda$, $\nu$ and $\mu$ such that $\lambda \ll \nu \ll \mu$.\\
	Then we have:
	$$\dfrac{\dlam}{\de\mu} =
	\dfrac {\de \lambda}{\de\nu} \dfrac{\de\nu}{\de\mu}.$$
\end{prop}
\begin{proof}
	For every $E \in \mm$ observe that:
	$$
		\lambda(E)
		= \int_E\frac{\dlam}{\de\nu} \,\de\nu
		= \int_E \frac{\dlam}{\de\nu}\frac{\de\nu}{\de\mu} \,\dmu
	;
	$$
	moreover, as $\lambda \ll \mu$, we have that:
	$$
		\lambda(E)
		= \int_E \frac{\dlam}{\de\mu} \de\mu
	,
	$$
	and then, on account of \vref{integral-inequality-sigma-finite}:
	$$
		\frac{\dlam}{\de\mu}
		= \frac{\dlam}{\de\nu}\frac{\de\nu}{\de\mu}
		\quad \text{a.e.}
	.
	$$
\end{proof}

\begin{prop}[Inverse derivative]
	If $\nu \ll \mu$ and $\mu \ll \nu$ then we have:
	$$\dfrac {\de\nu}{\de\mu} =
	\left(\dfrac{\de\mu}{\de\nu}\right)^{-1}.$$
\end{prop}
\begin{proof}
	For every $E \in \mm$ we have that:
	$$\mu(E) =\int_E \de\mu 
	= \int_E \frac{\de\mu}{\de\nu}\de\nu
	= \int_E \frac{\de\mu}{\de\nu}\frac{\de\nu}{\de\mu}\de\mu$$
	thus, always on account of \vref{integral-inequality-sigma-finite}:
	$$1 =
	\frac{\de\mu}{\de\nu}\frac{\de\nu}{\de\mu}
	\quad \text{a.e.}.$$
\end{proof}