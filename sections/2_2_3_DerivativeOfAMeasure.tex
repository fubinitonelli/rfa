%!TEX root = ../main.tex
\subsubsection{Derivative of a measure: definitions}\label{derivative-of-a-measure}

\paragraph{Measures defined through measurable functions} Here we see how a positive measurable function can provide an alternative measure.

\begin{theo}
	Let  $\phi: \Omega \to \left[0, +\infty\right]$ be a measurable function. Consider:
	$$\nu(E)\coloneqq\int_E \phi \,\de\mu \quad \text{for all } E \in \mm.$$
	Then, for any measurable function $f : \Omega \to \left[0, +\infty\right]$, $\nu(E)$ is a measure on $\mm$, and we have:
	$$\int_\Omega f \,\de\nu = \int_\Omega f \phi \,\de\mu.$$
\end{theo}

\begin{proof} \textit{Step 1, $\nu(E)$ is a measure on $\mm$}:\\
	It is easy to see that $\nu(\varnothing)=0 < +\infty$.\\
	Consider a sequence of mutually disjoint set $\{E_j\}_{j \in \NN} \subset\mm$ with $E=\bigcup_{j \in \NN}E_j$; using the monotone convergence for series (see corollary \vref{theo-beppo-series}), we have:
	\begin{align*}
		\nu(E) &= \int_E\phi \,\de\mu
		= \int_\Omega \phi \Ind_E \,\de\mu
		= \int_\Omega \sum_{j\in \NN}\phi \Ind_{E_j} \,\de\mu\\
		&= \sum_{j\in \NN} \int_\Omega \phi \Ind_{E_j} \,\de\mu
		= \sum_{j \in \NN} \int_{E_j} \phi \, \de \mu
		= \sum_{j\in \NN} \nu(E_j)
		;
	\end{align*}
	
	thus $\nu$ is countably additive.
	
	\textit{Step 2, exchange of measures}:\\
	Now let us focus on the second thesis of the theorem. It is enough to prove it when $f=\Ind_E$ with $E \in \mm$; then it can be easily extended to simple function and finally to positive functions by its approximation property. We have:
	$$
	\int_\Omega\Ind_E \,\de\nu 
	= \int_E d \nu 
	= \nu(E) 
	= \int_E \phi \,\de\mu 
	= \int_\Omega \phi \Ind_E \,\de\mu;$$
	that conclude the proof.
\end{proof}

In the first part of this proof we proved the following.
\begin{coro}
	Let $\phi$ be a positive and measurable function, and consider a sequence of disjoint sets $\{E_n\}_{n\in\NN}$. Then:
	$$\int_{\cup_{n \in \NN} E_n} \phi\, \de\mu = \sum_{n\in\NN}\int_{E_n} \phi \, \de\mu.$$
\end{coro}


\paragraph{Radon--Nikodym derivative} Suppose to have two measure, and we need to represent one in terms of the other. Possibly through an integral. This is possible by the followings.
\begin{defn}
	Consider a complete measure space $(\Omega,\mm,\mu)$ and a non-negative measurable function $\phi: \Omega \to [0,+\infty]$. If $\nu$ is such that:
	$$
		\nu(E) 
		= \int_E \phi \, \de \mu 
		\quad \text{for all }  E
	.
	$$
	Then $\phi$ is called \emph{Radon--Nikodym derivative} of $\nu$ with respect to $\mu$ and it is denoted by $\frac {\de\nu }{\de\mu}$.
\end{defn}
The integrand function is called ``derivative'' due to the analogy with the first fundamental theorem of calculus (see theorem \vref{theo-first-fundamental-calculus}) for which we have $F(x) = \int_a^x F'(t) \,\dt$.

For any measurable function $f:\Omega \to \left[0,+\infty\right]$ we can write:
\[
	\int_E f \,\de\nu =\int_E f \phi \,\de\mu = \int_E f \, \frac{\de\nu}{\de\mu} \,\de\mu
\]

Since we give the definition of Radon--Nikodym derivative, we ask ourselves if such function exists positive and measurable. We start observing that whenever the measure $\mu$ is zero on a set $E$, $\nu(E)$ must be zero too (otherwise we would have a null denominator and a non-null denominator). Indeed, if $\int_E f \,\de\nu = \int_E f \, \tfrac{\de\nu}{\de\mu} \,\de\mu$, then we require:
$$
	\mu(E) 
	= 0 
	\implies 
	\nu(E)
	= 0 
	\quad \text{for all } E \in \mm
.
$$
This motivates the following definition.

\begin{defn}
	We say that a measure $\nu$ is \emph{absolutely continuous} with respect to $\mu$, and we write $\nu \ll \mu$, if:
	$$
		\mu(E) 
		= 0 
		\implies \nu(E)
		= 0 
		\quad \text{for any } E \in \mm
	.
	$$
\end{defn}
In plain language, a measure $\nu$ is absolutely continuous with respect to $\mu$ if on all the sets for which $\mu$ is zero, $\nu$ is also zero.

This property holds when a measure can ``control'' the zero of another measure, it acquire a great meaning when considered with the previous definition: the measure $\mu$ controls the measure $\nu$ which was defined through $\mu$.

However, the condition $\nu \ll \mu$ is not sufficient alone for the existence of $\frac{\de\nu}{\de\mu}$; a sufficient condition for its existence in stated by the following theorem.

\begin{theo} [Radon--Nikodym] \label{theo-radon-nikodym}
	Let $(\Omega, \mm,\mu)$ be a complete measure space, and $\nu$, $\mu$ two measures on $(\Omega, \mm)$.\\
	If $\mu$ is $\sigma$-finite and $\nu\ll\mu$, then the Radon--Nikodym derivative $\frac{\de\nu}{\de\mu}$ exists.
\end{theo} 

See definition \vref{defn-positive-measure} for $\sigma$-finite. The theorem will be proved later on section \vref{proof-radon-nikodym}.

If $\mu$ is not $\sigma$-finite, then the existence of $\frac{\de\nu}{\de\mu}$ is not guaranteed.\\	
Indeed, consider the measure space $(\left[0,1\right],\Lc(\left[0,1\right]),\mu_c)$
where $\mu$ is the counting measure and let $\nu =\lambda\ll\mu_c$.
You can prove that there is no measure $\phi:\left[0,1\right] \to \left[0,+\infty\right]$ such that $\lambda(E) = \int_E \phi \, \de \mu_c \quad \forall E \in \Lc(\left[0,1]\right)$.
	
We will see that this is a powerful generalization of the second fundamental theorem of calculus (see theorem \vref{theo-2nd-found-calc}).
%\todo{Consider the following points}
%
%!!!! It is placed here (after fatou's lemma) because it is natural the possibility of define a measure from a measure using the integral
%
%!!!! It is an extension of the second fundamental theorem of calculus
%
%
%\begin{defn}
%	We say that $\mu$ is \emph{$\sigma$-finite} if there exists a sequence $\{E_j\}_{j \in \NN} \subset \mm$ such that:
%	$$\bigcup_{j\in\NN} E_j = \Omega \quad \text{and} \quad \mu(E_j) < +\infty \quad \forall j \in \NN$$
%\end{defn}
%
%\medskip
%\begin{exam}
%	\begin{enumerate}
%		\item The Lebesgue measure on $\RR^N$ is $\sigma$-finite;
%		\item the counting measure on $(\NN, \Pc(\NN)) $ is $\sigma$-finite;
%		\item the counting measure on $(\RR, \Lc(\RR)) $ is not $\sigma$-finite.
%	\end{enumerate}
%\end{exam}
%
%The following theorem states the sufficient conditions for the existence of $\frac{\de\nu}{\de\mu}$.
