%!TEX root = ../main.tex
\subsubsection{Linear compact operators}
Now we see a further property for linear operator. Here we require an improvement of the image of the operator that paves the way to new results.

Throughout this section, let $(X, \norm{\cdot}_X)$, $(Y, \norm{\cdot}_Y)$ be Banach spaces.
\begin{defn}
	A linear operator $K \in \Lc(X,Y)$, $K:X \to Y$ is \emph{compact} if for any bounded set $B \subseteq X$ the image $KB$ is precompact in $Y$, i.e. $\overline{KB}$ is compact.
	Their space is defined as $$\Kc(X,Y) \coloneqq \{ K \in \Lc(X,Y) \text{ compact} \}.$$
\end{defn}

The term precompact means that its closure is compact.

In other words, any bounded sequence $\{y_n\}_n \subseteq KB$ contains a converging subsequence which limit needs not to be in $KB$.

If $X=Y$, then often it is written as $\Kc(X)$.

\begin{prop}
	Any compact operator is bounded.
	\label{prop-compact-bdd}
\end{prop}
\begin{proof}
	Take the unit ball $B = B_1(0) \subseteq X$.\\
	As $KB$ is precompact, $\overline{KB}$ is compact, so it is is also bounded.\\
	Therefore there exists $M > 0$ such that:
	$$
		\norm K_\star 
		= \sup\limits_{x \in \widebar B} \norm{Kx} 
		\le M
	$$ 
	that is, $K \in \Bc(X,Y)$.
\end{proof}

\begin{defn}
	A linear operator is \emph{finite-rank} if its image is finite dimensional. 
\end{defn}

\begin{prop}
	A finite-rank bounded operator is also compact. 	
\end{prop}
Indeed, notice that if $T \in \Bc(X,Y)$ and $Y$ is finite-dimensional then $T$ is compact.\\
However, there are compact operators which are not finite-rank: consider $(C([a,b]), \norm{\cdot}_\infty)$ and, for each $u \in C([a,b])$, set:
$$Ku(t) = \int_a^t u(s) \ds \quad \forall t \in [a,b].$$
The linear operator $K: C([a,b]) \to C([a,b])$ is compact (see Ascoli--Arzelà theorem \vref{theo-ascoli-arzela}), indeed we have $\Im(K)=\{f \in \Cc^1([a,b]): f(a) = 0\}$.

Note also that if $X$ and $Y$ are infinite-dimensional, then $\Kc(X,Y) \subsetneq \Bc(X,Y)$. \\ 
An example of bounded, non-compact operator is the identity map, if $X=Y$ infinite-dimensional.

%\begin{rema}
%	In terms of sequences: $\forall \{ x_n \} \subset X$ bounded $\exists \, \{x_{n_h}\}$ such that $\{ K x_{n_h}\}$ converges to some $y \in Y$
%\end{rema}

\begin{defn}
	If $Y$ is a subspace of $X$ and the identity $I : Y \to X$ is compact we say that $Y$ is \emph{compactly embedded} in $X$ and we write: $$Y \hookrightarrow\hookrightarrow X.$$
\end{defn}

\begin{prop}
	Let $X$ be an infinite dimensional Banach space.\\
	Then a compact operator $K:X \to X$ cannot be bijective.
\end{prop}

\begin{proof}
	By contradiction, take $\{y_n\}_{n \in \NN} \subset B$, where $B$ is the unit ball of $X$.\\
	Consider now $x_n = K^{-1} y_n$: the sequence $\{x_n\}_{n \in \NN}$ is bounded, as $K^{-1}$ is bounded.\\
	Since $\{y_n\}_{n \in \NN}$ contains a convergent sub-sequence, $B$ is compact, which is a contradiction to \vref{theorem-riesz}.
\end{proof}


\paragraph{Characterization for compact operators} To prove directly the compactness is not an easy task, anyway we can find a criterion to do the job.

\begin{theo}[Compactness characterization] \label{theo-compact-charact}
	If $K \in \Kc(X,Y)$ and $x_n \wto  x$, then: $$ K x_n \to Kx.$$
	This means that $K$ is \emph{weak-strong continuous}.
	
	Moreover, if $X$ is reflexive and $K \in \Bc(X,Y)$ is weak-strong continuous, then $K$ is also compact.
\end{theo}


\begin{proof} The two parts will be proved separately.\\
	\textit{Proof of the first part:}\\ 
	Suppose $K \in \Kc(X,Y)$ and $x_n \wto x$. Being also bounded (see \vref{prop-compact-bdd}), $K$ is weak-weak continuous (see \vref{prop-bdd-weak-weak}), namely $Kx_n \wto Kx$.\\
	If $\{x_n\}$ is bounded, we have that $\{Kx_n\}$ is also bounded and contains a strongly convergent subsequence, and $\{Kx_n\}$ has a non-empty and bounded class limit.
	
	Let $y$ be a limit point: there exists $\{K x_{n_h}\}_{h \in \NN}$ such that $Kx_{n_h} \to y$ and in particular $Kx_{n_h} \wto y$.\\
	Therefore $y = Kx$ is the only limit point: we have that $K$ is weak-strong continuous, namely $Kx_n \to Kx$.
	
	\textit{Proof of the second part:}\\ 
	Suppose now that $K \in \Bc(X,Y)$ is weak-strong continuous, consider a bounded set $E \subset X$  and the sequence $\{y_n\}_{n\in\NN} \subset KE$.\\
	Then there exists $x_n \in E$ such that $Kx_n = y_n$ for every $n \in \NN$.\\
	
	Using the reflexivity of $X$ we can apply the corollary \vref{prop-coro-BA} to the BA theorem\footnote{Remember that separability was not necessary, even though to prove it without this assumption, we can't take advantage of the Banach-Alaoglu theorem.} and find $\{x_{n_h}\}_{h \in \NN}$ such that $x_{n_h}\wto x$. Hence $Kx_{n_h} \to Kx$ and $KE$ is precompact. This entails $K \in \Kc(X,Y)$.
\end{proof}

We have seen that $\Kc(X,Y)$ is a subspace of $\Bc(X,Y)$. We are now going to prove that $\Kc(X,Y)$ is closed.

\begin{theo}
	The space of compact operators $\Kc(X,Y)$ is a closed subspace of $\Bc(X,Y)$.\\
	This means that $\Kc(X,Y)$ is a Banach space with respect to the induced norm.
\end{theo}
\begin{proof} \textit{For simplicity, we will prove the theorem when $X$ is also reflexive.}\\
	Consider a converging sequence $\{K_n\}_{n\in\NN} \subset \Kc(X,Y)$, such that there exists: $$K \in \Bc(X,Y) : \enspace \norm{K_n-K}_{\Bc(X,Y)}\to 0 \quad \text{as } n \to +\infty.$$
	We are left to prove that $K$ is compact, namely $K \in \Kc(X,Y)$, by showing that it is weak-strong continuous (see the theorem stating the characterization of the compactness \vref{theo-compact-charact}).
	
	Let $\{x_n\}$ such that $x_n \wto x$; as $\{x_n\}_{n\in\NN}$ is bounded there exists $M > 0$ such that $\norm{x_n}_X \leq M$, and we have:
	\begin{align*}
		\norm{Kx_n - Kx}_Y 
		& \leq \norm{Kx_n-K_j x_n}_Y + \norm{K_jx_n- K_j x}_Y + \norm{K_j x- K x}_Y\\
		& \leq 2M \norm{K-K_j}_{\Bc(X,Y)} + \norm{K_jx_n- K_j x}_Y
	\end{align*}
	Let $\eps > 0$ be fixed. Then exists $j_0 \in \NN$ such that $$\norm{K_j -K}_{\Bc(X,Y)} < \frac{\eps}{3M} \quad \forall j \geq j_0.$$
	Therefore:
	$$\norm{Kx_n - K_j x_n}_Y 
	\leq M \cdot \norm{K-K_j}_{\Bc(X,Y)} < \tfrac \eps 3
	\quad
	\forall j \geq j_0
	.$$
	Moreover, since $x_n \wto x$, via lower semi-continuity we have:
	$$
	\norm{K_jx-Kx}_Y 
	\leq \liminf_n \norm{K_j x_n-Kx_n}_Y \le \tfrac \eps 3
	\quad
	\forall j \geq j_0
	.$$
	Fix $j=j_0$. Since $K_{j_0} \in \Kc(X,Y)$, there exists $n_0 \in \NN$ such that:
	$$
	\norm{K_{j_0}x_n - K_{j_0}x}_Y 
	< \tfrac \eps 3
	\quad
	\forall n \geq  n_0
	.$$
	
	Summing up:
	$$\forall \eps > 0 \quad \exists \, n_0=n_0(\eps) \in \NN : \enspace \norm{Kx_n -Kx}_y < \eps \quad \forall n \geq n_0$$
	Thus $Kx_n \spaceto{Y} Kx$, and $K$ is compact.
\end{proof}

\paragraph{The approximation problem} The previous theorem has an immediate consequence:
\begin{prop}
	If we have a converging sequence of finite-rank linear  operators, then its limit is a compact operator.
\end{prop}
The converse of this proposition is known as \emph{approximation property}: is any $K \in \Kc(X,Y)$ the limit of a sequence of finite rank operators, with respect to the operator norm? 

In general it is not: Per Enflo proved in 1973 that there exists a Banach space which is separable but it does not have any Schauder basis and so the approximation property does not hold.

We know that the approximation property holds if $Y$ has a Schauder basis or if $Y$ is a Hilbert spaces (see definition \vref{defn-hilbert-spaces}).

%neumann series, min 8 of lesson 30/11/2020

\paragraph{First example of compact operator} We are now going to introduce a class of linear compact operators, from $L^p$ to $L^q$.\\
Consider $(\Omega, \Lc(\Omega), \lambda)$ with $\Omega \in \Lc(\RR^N)$
and $G \in L^q(\Omega \times \Omega)$ with respect to the Lebesgue measure in $\RR^{2N}$
for some $q \in (1, \infty)$.

For any $u \in L^p(\Omega)$, where $p$ is the conjugate of $q$, set
$$K_G u(x) = \int_\Omega G(x,y) u(y) \,\dy$$
for all $u \in L^p(\Omega)$ and for almost any $x \in \Omega$.
The function $G$ is called \emph{kernel} of the operator $K_G$.

Using Hölder's inequality we have:
$$
	|K_G u(x)|^q
	\leq \left( \int_\Omega |G(x,y)| \ |u(y)| \,\dy \right)^q
	\leq \norm{G(x, \cdot)}_q^q\norm{u}_p^q
$$
for almost any $x \in \Omega$. This implies that:
$$
	\norm{K_G}_{q}
	\leq \norm{G}_{q} \norm{u}_{p}
	\quad \forall u \in L^p(\Omega)
	.
$$
Then we deduce that $K_G$ is linear and bounded from $X=L^p(\Omega)$ and $Y=L^q(\Omega)$.

%Let us prove that $K_G \in \Kc(X)$; we can use the characterization given by \vref{theo-compact-charact}. %è certamente cannata ma può avere un suo perché
Let $\{u_n\}_{n\in \NN} \subset L^p(\Omega)$ be such that $u_n \wto u$.\\
Define the converging sequence:
$$\Phi_n(x) = K_G(u_n - u)(x) = \int_\Omega G(x,y)(u_n - u)(y) \, \dy \to 0$$
for almost any $x \in \Omega$. Notice that we have:
$$|\Phi_n(x)|^q \leq \norm{G(x, \cdot)}^q_q \norm{u_n - u}_q^q \leq M \norm{G(x, \cdot)}_q^q  = F(x)$$
for almost any $x \in \Omega$ since $\{u_n\}_{n \in \NN}$ is bounded in $L^p(\Omega)$.

On the other hand, $F \in L^1(\Omega)$. Therefore (see dominated convergence theorem \vref{dominated-convergence}):
$$ \norm{K_G(u_n -u)}_q \to 0 \text{ as } n \to \infty.$$
Hence $K$ is weak-strong continuous so it's compact since $L^p(\Omega)$ is reflexive.

In case of $p=q=2$ we have the following definition.

\begin{defn}
	Let $G \in L^2(\Omega \times \Omega)$, for all $u \in L^p(\Omega)$ the operator
	$$K_G u(x) = \int_\Omega G(x,y) u(y) \,\dy$$
	is called \emph{Hilbert--Schmidt} operator with kernel $G$.
	
	The set of all Hilbert--Schmidt operators is a subspace of $\Kc(L^2(\Omega))$.
\end{defn}

\paragraph{Second example of compact operator}
Let's set: 
$$X_p = \{f \in AC([a,b]): \ f' \in L^p((a,b))\} 
\subseteq AC([a,b])$$
where $p \in [1,\infty]$. We know that $X_p$ can be identified with some \textit{Sobolev space}
$$W^{1,p}((a,b)) \coloneqq \{f \in L^p([a,b]) : \ Df \in L^p ((a,b))  \}$$
where $Df$ is the distributional derivative, namely:
$$\int_a^b Df \phi \, \dt =
-\int_a^b f \phi' \, \dt
\quad \phi \in \Cc_C^\infty(((a,b)).$$

In particular we have that $AC([a,b])$ can be identified with $W_1^1((a,b))$ since each equivalence class contains one and only one continuous representative.

It's easy to prove that $W^{1,p}(a,b)$ is a Banach space with respect to the norm:
$$\norm{f}_{\spadesuit, p} = |f(a)| + \norm{Df}_p$$
which is equivalent to 
$$\norm{f}_{1, p} = \norm{f(a)}_{p} + \norm{Df}_p.$$

Consider now the identity application:
$$I: (W^{1,p}((a,b)), \norm{\cdot}_{\spadesuit, p}) \to (\Cc([a,b]), \norm{\cdot}_\infty), \quad I f=f.$$

Using Ascoli--Arzelà theorem (\vref{theo-ascoli-arzela}) we can prove that $I$ is compact if $p \in (1,\infty]$.
Thus we have
$$W^{1,p} ((a,b)) \hookrightarrow \hookrightarrow \Cc([a,b]) \quad \forall p > 1.$$

Let $\{f_n\}_{n \in \NN} \subset W^{1,p}((a,b))$ be bounded by a constant $M > 0$.
Observe that
$$f_n(t)= f_n(a) + \int_a^t Df_n(r) \, \dr \quad t \in [a,b].$$

Then, using Hölder's inequality, we get
\begin{align*}
	|f_n(t)| & \geq |f_n(a)| + \int_a^b |Df_n(r)| \, \dr \\
	& \geq |f_n(a)| + (b - a)^{\frac 1 q} \norm{Df_n}_p\\
	& \max\{1, (b-a)^{\frac 1 q} \}\norm{f_n}_{\spadesuit,p}
\end{align*}
where $q \in [1, \infty)$ is the conjugate of $p$. Thus we find:
$$\norm{f_n}_\infty \leq \max \{1, (b-a)^{\frac 1 q}\}M \quad \forall n \in \NN.$$

Observe now that 
$$f_n(t)- f_n(s) = \int_s^t Df_n(r)\, \dr \quad t,s \in [a,b].$$

Using again Hölder's inequality we find:
\begin{align*}
	|f_n(t) - f_n(s)| &\leq |\int_s^t |Df_n(r) \, \dr|\\
	& \leq |t-s|^{\frac 1 q} \norm{Df_n}_{p} \\
	& \leq M |t-s|^{\frac 1 q}.
\end{align*}

Then $\{f_n\}_n\in\NN$ is bounded and equicontinuous. Using Ascoli--Arzelà theorem we deduce that there exists a subsequence $I f_{n_h} = f_{n_h}$ for $h \in \NN$ and $f\in C([a,b])$ such that $\norm{f_{n_h}-f}_\infty \to 0$ as $h \to \infty$. Thus $I$ is compact.

\paragraph{The Fredholm alternative in Banach spaces}
Surjectivity and injectivity of an operator are strongly related to the solvability of a certain equation somehow associated to the operator itself. The following results will handle a scenario where the operator $I-K$ for some $K$ compact.

\begin{theo}
	Let $X$ be a Banach space and $K \in \Kc(X)$. Then 
	\begin{itemize}
		\item $\Ker(I-K)$ is finite dimensional;
		\item $\Im(I-K)$ is closed;
		\item $\Ker(I-K) = \{0\}$ if and only if $\Im(I-K) =X$.
	\end{itemize}
\end{theo}
This result will be proven for Hilbert spaces (see \vref{theo-fredholm-h-spaces}).

\medskip
\begin{coro}[Fredholm's alternative]
	\label{coro-fredholm-b-spaces}
	Let $K \in \Kc(x)$. Given $y \in X$, consider the functional equation:
	$$
		x-Kx=y
	$$
	Then either the equation has a unique solution for any $y \in Y$ or $x-Kx=0$ has at $n>1$ linearly independent solutions.
\end{coro}

\begin{coro}
	Let $X$ finite dimensional and $K \in \Kc(X)$.\\
	Then the operator $K$ is injective if and only if it is also surjective.
\end{coro}

Otherwise if $X$ is infinite-dimensional then there are linear operator which are injective but non surjective or viceversa.
\begin{exam}
		Consider $X=l^2$ and $x= \{x_1,x_2,\ldots,x_n,\ldots\} \in X$, take:
		$$
		Rx = \{0, x_1, x_2, \ldots, x_n, \ldots\}, \qquad Lx = \{x_2,x_3, \ldots, x_n, \ldots\}.
		$$
		We have that $R$, the right shift, is injective but not surjective while $L$, the left shift, is surjective but not injective.
\end{exam}

%
%\begin{exam} Let us apply the theorem to Hilbert--Schmidt operators.
%	
%	Consider $K_G \in \Kc(x)$ where $X=L^2([0,1])$. Suppose that $\norm{G}_{L^2([0,1]^2)} < 1$. If there exists $u \in X$ such that $u-K_G u = 0$, then $u=0$. 
%	
%	Indeed, if $u=K_Gu$, then:
%	$$\norm{u}_{L^2([0,1])} = \norm{K_G u}_{L^2([0,1])} \leq \norm{G}_{L^2([0,1])}\norm{u}_{L^2([0,1])} < \norm{u}_{L^2([0,1])}$$
%	Thus $\norm{u}_{L^2([0,1])} = 0$ and $u = 0$.
%	
%	Therefore $\Ker(I-K_G) = \{0\}$. The second alternative is impossible, and thus $\forall v \in X, \exists! n \in X$ such that $n-K_G u = v$
%\end{exam}	
%
%\medskip
%\begin{exer}
%	Set $G(X,Y) = \frac{\Lambda}{(X-Y)^\alpha}$ with $\alpha>0$ and $\lambda \neq 0$. Find $\lambda,\ \alpha$ such that $\norm{G}_{L^2([0,1])}<1$.
%\end{exer}
%
%\medskip
%\begin{exer}
%	Show that any $K \in \Kc(x)$ cannot be both injective and surjective if $X$ is infinite-dimensional.
%\end{exer}
%
%\begin{coro}
%	Let $K \in \Kc(X)$. $I-K \in \Bc(X)$ is injective if and only if it is also surjective.
%\end{coro}
%
%If the operator is only bounded, then this result does not hold in general.
%\begin{exam}
%	Consider $X=l^2$ and take:
%	\begin{align*}
%	T_\Omega(x_1,x_2, \ldots, x_n, \ldots) &= (0, x_1, x_2, \ldots, x_n, \ldots) \\
%	T_l(x_1,x_2,\ldots,x_n,\ldots) &= (x_2,x_3, \ldots, x_n, \ldots)
%	\end{align*}
%\end{exam}
%$T_\Omega$ is injective but not surjective. $T_l$ is surjective but not injective.
