%!TEX root = ../main.tex
\subsubsection{Limits and continuity for functions}
With the tools that we have been developed, now we face the definition of a notion of continuity for functions (recall the definition \vref{function}). A general notion for continuous functions is the following:
\begin{defn}\label{continuous-functions-general}
	A function said to be is continuous if it transforms open sets into open sets.
\end{defn} 
A formal definition need to specify on which structure we are working, for which we develop the two different cases in the followings; as we are going to see, the definition of limits differs and the definition of continuity is strictly related to the definition of limit. For metric spaces see theorem \vref{continuous-functions-general-metric}, for topological spaces see theorem \vref{continuous-functions-general-topological}.
We could also equivalently define continuous functions as the ones mapping closed sets in closed sets, but using open sets theory is simpler.

\paragraph{Limits and continuity for functions in metric spaces}
First, we develop and study the continuity in the context of metric spaces. The following definition is similar to \vref{limit-in-metric-spaces} where we defined the limit for sequences, now we extend that definition to functions in general.
\begin{defn} \label{limit-function-in-metric-spaces}
	Let $(X, d_X)$, $(Y, d_Y)$ be metric spaces and consider a function $f: X \to Y$.\\
	Then the \emph{limit} of $f(x)$ is $y_0$, namely $f(x) \to y_0$, as $x \to x_0$, with $x_0$ accumulation point for $X$, if:
	$$
		\forall \varepsilon > 0 \quad \exists \, \delta = \delta(x_0, \eps) > 0: \ d_X(x, x_0) < \delta \implies d_Y(f(x), y_0) < \eps$$
	or, equivalently using open balls:
	$$\forall \varepsilon > 0 \quad \exists \, \delta = \delta(x_0, \eps) > 0: \ f(B_{d_X}(x_0, \delta) \setminus \{x_0\})\subset B_{d_Y}(y_0, \eps).$$
	When such limit does exist, then we say that the function \emph{converges} to $y_0$ in $x_0$, using the following notation:
	$$ \lim\limits_{x \to x_0}f(x)=y_0.$$
\end{defn}
As for sequences, the limit, if exists, is unique.

Now we can provide the ``classical'' definition of continuous function:
\begin{defn} \label{continuous-functions-metric}
	Let $(X, d_X)$, $(Y, d_Y)$ be metric spaces and consider a function $f: X \to Y$.
	
	We say that $f$ if \emph{continuous} in $x_0$ if one of the following condition is satisfied:
	\begin{itemize}
		\item $x_0$ is an accumulation point of $X$ and $f(x) \to f(x_0)$ as $x \to x_0$;
		\item $x_0$ is an isolated point for $X$.
	\end{itemize}
	
	We say that $f$ is \emph{continuous in a set} $D \in X$ if it is continuous in every point of $D$.
\end{defn}

A stronger version of continuity is the following:
\begin{defn}\label{uniformly-continuous-functions}
	Let $(X, d_X)$, $(Y, d_Y)$ be metric spaces and consider a function $f: X \to Y$.\\
	We say that $f$ is \emph{uniformly continuous} in $D \subset X$ if:
	$$
		\forall \varepsilon > 0 
		\quad \exists \, \delta = \delta(\eps) > 0 
		\quad \forall x, y \in D : 
		\ d_X(x, x_0) < \delta 
		\implies d_Y(f(x), f(y)) 
		< \eps
	.
	$$
\end{defn}
Notice that in this definition $\delta$ does not depend on $x$, but only on $\eps$.
\begin{prop}
	Let $(X, d_X)$, $(Y, d_Y)$ be metric spaces and consider a function $f: X \to Y$.\\
	If $f$ is uniformly continuous in $D$, then $f$ is continuous in $D$. The converse isn't true in general.
\end{prop}


The following is a particular definition of continuity that occurs when functions converges like sequences in their domain:
\begin{defn} \label{defn-seq-continuity-metric-sp}
	Let $(X, d_X)$, $(Y, d_Y)$ be metric spaces and consider a function $f: X \to Y$.\\
	We say that $f$ is \emph{sequentially continuous} in $x_0$ if:
	$$\forall \{x_n\} \subset X \quad x_n \to x_0 \implies f(x_n) \to f(x_0).$$
\end{defn}

\begin{prop}
	Let $(X, d_X)$, $(Y, d_Y)$ be metric spaces and consider a function $f: X \to Y$.\\
	The function $f$ is continuous in $x_0$ if and only if $f$ is sequentially continuous in $x_0$.
\end{prop}

\begin{proof}
	If $x_0$ is an isolated point the thesis is trivial. Otherwise, let $x_0$ be an accumulation point.
	
	\textit{Necessary condition} $\implies$:\\
	Consider a sequence $\{x_n\} \subseteq X$ such that $x_n \to x_0$ and fix $\eps > 0$.
	
	We gain two consequences, $f$ is continuous at $x_0$, namely: 
	$$
		\forall \varepsilon > 0 
		\ \exists \, \delta > 0:
		\ 0<d_X(x_n, x_0) < \delta 
		\implies d_Y(f(x), f(x_0)) < \eps
	$$ 
	and that $x_n$ converges to $x$, namely: 
	$$\exists \, \bar{n} > 0: \ \forall n > \bar{n} \implies d_X(x_n,x_0) < \delta.$$
	
	Then if $n > \bar{n}$ we have $d_Y(f(x_n),f(x_0)) < \eps$, that is $f(x_n) \to f(x_0)$: this prove the implication.
	
	\textit{Sufficient condition} $\impliedby$:\\
	By contradiction, if $f$ isn't continuous in $x_0$ then:
	\[
		\exists \, \bar\eps > 0 :
		\ \forall \delta > 0 
		\ \exists \, x_\delta \in X :
		\ 0<d_X(x_j, x_0) < \delta
		\quad \text{and} 
		\quad d_Y(f(x_\delta), f(x_0)) \geq \bar\eps
	. 
	\tag{$\star$}
	\]
	
	Let $\delta_n = \frac 1 n$, with $n\in\NN$ and let $x_n = x_{\delta_n}$. Then $\{x_n\}\subseteq X$ and $d_X(x_n,x_0)<\frac 1 n$ meaning that $x_n \to x_0$.
	
	By $(\star)$ we have $d_Y(f(x_n),f(x_0))\geq \bar{\eps}$, that is $f(x_n)$ doesn't converges to $f(x_0)$ so $f$ is not sequentially continuous. This prove the co-implication.
\end{proof}

The general definition for continuity \vref{continuous-functions-general} can be written as follows in terms of metric spaces:
\begin{theo}[Characterization of continuity in metric spaces]\label{continuous-functions-general-metric}
	Let $(X, d_X)$, $(Y, d_Y)$ be metric spaces and consider a function $f: X \to Y$.\\
	Then the function $f$ is continuous in $X$ if $\forall A \subset Y$ open set we have that $f^{-1}(A) \subset X$ is an open set.
\end{theo} 

The continuity is invariant with respect to equivalent distances:
\begin{prop}
	Let $(X, d_1)$, $(X, d_2)$ be metric spaces, where $d_1$ is equivalent to $d_2$. \\
	A function $f: (X, d_1) \to (X, d_1)$ is continuous if and only if $f: (X, d_2) \to (X, d_2)$ is also continuous.
\end{prop}

\begin{prop}
	Let $(X, d_X)$, $(X, \tilde d_X)$ be metric spaces, such that $\exists\, C_1 > 0 : d_x \le C_1 \tilde d_x$. \\
	Let also $(Y, d_Y)$, $(Y, \tilde d_Y)$ be metric spaces, such that $\exists\, C_2 > 0 : \tilde d_y \le C_2 d_y$.\\
	If $f: (X, d_X) \to (Y, d_Y)$ is continuous, then $f: (X, \tilde d_X) \to (Y, \tilde d_Y)$ is also continuous.
\end{prop}

\begin{proof}
	Let $x_0$ in $X$. By hypothesis, we have:
	$$\forall \eps > 0 \quad \exists \, \delta = \delta(x_0, \eps) : \ d_X(x, x_0) < \delta \implies d_Y(f(x), f(x_0)) < \eps$$
	Take $\tilde\eps > 0$ and define $\tilde\delta(\tilde\eps) \coloneqq \frac{1}{C_1} \cdot \delta \left( \frac{\tilde\eps}{C_2} \right)$. Then we have:
	\begin{align*}
	\tilde d_X(x, x_0) < \tilde\delta(\tilde\eps)
	&\implies d_X(x, x_0) < \delta \left( \tfrac{\tilde\eps}{C_2} \right) \\
	&\implies d_Y(f(x), f(x_0)) < \tfrac{\tilde\eps}{C_2} \\
	&\implies \tilde d_Y(f(x), f(x_0)) < \tilde\eps
	\end{align*}
	Because $\tilde\eps$ is arbitrary, $f: (X, \tilde d_X) \to (Y, \tilde d_Y)$ is continuous by definition.
\end{proof}

\paragraph{Limits and continuity for functions in topological spaces} The notion of continuity function is slightly different.
\begin{defn}
	Let $(X, \tau_X)$, $(Y, \tau_Y)$ be topological spaces, and consider a function $f: X \to Y$.\\
	Then the \emph{limit} of $f(x)$ is $y_0$, namely $f(x) \to y_0$, as $x \to x_0$, with $x_0$ accumulation point for $X$, if:
	$$\forall B \in \tau_Y : y_0 \in B \quad \exists \, A \in \tau_X : x \in A, \, f(A \setminus \{x_0\}) \subset B.$$
	When such limit does exist, then we say that the function \emph{converges} to $y_0$ in $x_0$, using the following notation:
	$$ \lim\limits_{x \to x_0}f(x)=y_0.$$
\end{defn}
As for sequences in topological spaces, and as opposed to the metric space definition, the limit can be not unique.

The definitions of continuity and sequential continuity are the same ones already given for metric spaces (see definitions \vref{continuous-functions-metric} and \vref{defn-seq-continuity-metric-sp}), simply with a different notion of convergence. Anyway, there is a difference in their implications:

\begin{prop}
	Let $(X, \tau_X)$, $(Y, \tau_Y)$ be topological spaces, and consider a function $f: X \to Y$.\\
	If $f$ is continuous in $x_0$, then $f$ is sequentially continuous in $x_0$.
\end{prop}
The converse is not true in general, as topological spaces cannot be described by sequences. But, if $(X, \tau_X)$ is a first countable topological spaces (see \vref{first-second-countable}), then the double implication holds.

As we done in \vref{continuous-functions-general-metric}, we will now provide another version of the definition of continuity based on open sets (see \vref{continuous-functions-general}), version adequate to the context of topological spaces:

\begin{theo}[Characterization of the continuity in topological spaces]\label{continuous-functions-general-topological}
	Let $(X, \tau_X)$, $(Y, \tau_Y)$ be topological spaces, and consider a function $f: X \to Y$.\\
	Then the function $f$ is \emph{continuous} in $X$ if:
	$$
		\forall A \in \tau_Y 
		\quad \text{we have} 
		\quad f^{-1}(A) \in \tau_X
	.
	$$
\end{theo} 
