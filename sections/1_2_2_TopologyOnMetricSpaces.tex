%!TEX root = ../main.tex
\subsubsection{Topological structure and sequences on metric spaces}

Thanks to the notion of metric and the definition of balls we can now build a topological structure on metric spaces:
\begin{defn} \label{topological-structure-metric-spaces}
	Let $\left(X,d\right)$ be a metric space, $A\subset X$, $x_0 \in X$. We say that:
	\begin{itemize}
		\item $x_0$ is an \emph{interior point} of $A$ if there exists $r > 0$:	
			$$
			B_r\left(x_0\right)\subset A
			;
			$$%\text{ and }B_d\left(x,y\right)=B_r\left(x\right);$$
		\item $x_0$ is an \emph{exterior point} of $A$ if there exists $r>0$:
			$$
			B_r\left(x_0\right)
			\subset \left(X \setminus A\right) 
			= A\comp
			;
			$$
		\item $x_0$ is a \emph{boundary point} of $A$ if it is neither interior nor exterior;
		\item $x_0$ is a \emph{adherent point} of $A$ if for all $r>0$:
			$$
			B_r\left(x\right)\cap A 
			\neq \varnothing
			$$
			so any point of $A$ is an adherence point of $A$;
		\item $x_0$ is an \emph{accumulation point} (or limit point or cluster point) of $A$ if for all $r>0$ exists $x_n$:
			$$
			x_n 
			\in \left\{B_r (x_0) \cap A\right\}\setminus\{x_0\}
			;
			$$
		\item $x_0$ is an \emph{isolated point} of $A$ if:
			$$
			x_0 
			\in A
			\text{ and }
			\exists \, r>0 : 
			B_r\left(x_0\right)\cap A
			=\{x_0\}
			.
			$$
	\end{itemize}
\end{defn}

Moreover, the set of each ``kind'' of point can be identified with a proper name:
\begin{defn} \label{notable-set-portions}
	Let $\left(X,d\right)$ be a metric space, $A\subset X$, $x_0 \in X$, than:
	\begin{itemize}
		\item the \emph{interior} set of $A$ is defined as:
					$$\mathring{A} = \text{int}\left(A\right) \coloneqq \{x_0 \in A : x_0 \text{ is an interior point of A}\};$$
		\item the \emph{exterior} set of $A$ is defined as:
					$$\ext\left(A\right) \coloneqq \{x_0 \in A: x_0\text{ is an exterior point of }A\};$$
		\item the \emph{boundary} of $A$ is defined as:
					$$\partial A \coloneqq \{x_0 \in A: x_0\text{ is a boundary point of }A\};$$
		\item the \emph{closure} of $A$ is defined as:
					$$\bar{A} \coloneqq \{x_0 \in A: x_0\text{ is an adherent point of }A\};$$
		\item the \emph{derived} set of $A$ is defined as:
					$$A'\coloneqq \{x_0 \in X: x_0\text{ is an accumulation point of }A\}.$$
	\end{itemize}
\end{defn}

This allow us to specify whether a set is open or closed:
\begin{defn}\label{open-close-set-metric-spaces}
	The set $A$ is \emph{open} if $A=\mathring{A}$; that is every $x\in A$ is an interior point.\\
	The set $A$ is \emph{closed} if $X\setminus A$ (or $A\comp$) is open.
\end{defn}

With these new concepts we can extend the notion of disjoint sets (see definition \vref{main-set-operations}):
\begin{defn}
	We say that two sets, $A$ and $B$, are \emph{almost disjoint}, or disjoint except for boundaries, if $$\mathring{A}\cap \mathring{B} = \varnothing.$$
\end{defn}

Now we highlight some basic properties.
\begin{prop}[Topological properties of sets] \label{topological-properties-of-sets}
	Let $A \subset X$, then:
	\begin{itemize}
		\item the sets $\mathring{A}$, $\partial A$ and $\text{ext}\left(A\right)$ are a partition of $X$;
		\item $A$ is open $\iff A=\mathring{A} \iff A \cap \partial A = \varnothing$;
		\item $\bar{A}= A \cup \partial A = \mathring{A} \cup \partial A$;
		\item $A$ is closed $\iff A=\bar{A} \iff \partial A \subseteq A$;
		\item $A'=\bar{A}\setminus \{\text{isolated points of A}\}$;
		\item $\mathring{A}$ is the largest open set containing $A$;
		\item $\bar{A}$ is the smallest closed subset of $X$ containing $A$.
	\end{itemize}
\end{prop}

\begin{prop}[Topological property for family of sets] \label{topological-properties-of-families-of-sets}
	Now, let $I$ be a family of indexes of any cardinality (countable or uncountable) and $\{A_j\}_{j \in I} \subset \Pc(X)$ and  $m \in I$:
	\begin{itemize}
		\item if $A_j$ is a open set $\forall j \in I$, then $\bigcup_{j\in I} A_j$ is open;
		\item if $A_1, \ldots, A_m$ are open sets, then $\bigcap_{j=1}^m A_j$ is open;
		\item if $A_j$ is a closed set $\forall j \in I$, then $\bigcap_{j\in I} A_j$ is closed;
		\item if $A_1, \ldots, A_m$ are closed sets, then $\bigcup_{j=1}^m A_j$ is closed.
	\end{itemize}
\end{prop}
So uncountable union of open set is an open set while uncountable intersection of closed set is a closed set.\\
Notice that the first two properties are the complementary to the last two due to De Morgan's law.

\paragraph{Sequences in metric spaces} Much of the previous definition about distances allow us to study the sequences defined in metric spaces, in particular we now discuss about their convergence. A sequence is, formally, a function which domain is $\NN$ (see definition \vref{function}). In a less formal language, a sequence can be considered an ordered set of elements which are indexed on $\NN$. We will denote a sequence as follows:
$$
\{x_n\}_{n \in \NN}
.
$$
During theory deployment we will be interested in discover the behavior of a given sequence, for example if it converges or not. Here we start crafting tools useful for such goal.
\begin{defn} \label{limit-in-metric-spaces}
	Let $(X, d)$ be a metric space, $\{x_n\}$ be a sequence in $X$, $x^\star \in X$.\\
	Then the \emph{limit} of the sequence $\{x_n\}$ is $x^\star$, namely $x_n\to x^\star$, as $n\to \infty$ if:
	$$\forall\varepsilon > 0 \quad \exists \, \bar{n}=\bar{n}\left(\varepsilon\right): \ n>\bar{n}\; \implies\; d\left(x_n,x^\star\right)<\varepsilon$$
	or, equivalently using open balls:
	$$\forall\varepsilon > 0 \quad \exists \, \bar{n}=\bar{n}\left(\varepsilon\right): \ n>\bar{n}\; \implies\; x_n\in B_\varepsilon \left(x^\star\right).$$
	When such limit does exist, then the sequence is said to be \emph{convergent}.
\end{defn}

From this definition comes three remarks: first, the limit is unique and any sub-sequence $\{x_{n_k}\}$ converges to the same limit: to prove this use the triangular inequality and the identity of indiscernibles. Second, considering two different but equivalent metrics the limit is the same, namely let $d_1, d_2$ be equivalent distances on set $X$, then $x_n \xrightarrow{d_1} x^* \iff x_n \xrightarrow{d_2} x^*$. Third, there is a relation between a set and the sequences which it contains, in particular we have the following theorem:

\begin{theo}[theorem closure in metric spaces]\label{theo-closure-metric-spaces}
	Let $(X,d)$ be a metric space and consider $A \subset X$. \\
	The subset $A$ is closed if and only if, for any sequence $\{x_n\} \subset A$, we have that: 
	$$
		x_n \to x 
		\implies x \in A
	.
	$$
\end{theo}
This theorem provides a characterization for closedness in metric space. The condition highlighted is known as \textit{sequentially closedness} and in case of metric space is completely equivalent to the closedness (see definition \vref{sequentially-closed-spaces-proposition}, and following results). We will see that this is not true for topological spaces.

\begin{proof}[Necessary condition $(\implies)$]:\\
	Having that $A$ is closed and any sequence $\{x_n\}$ such that $x_n \to x$, we have to prove that $x \in A$.\\
	By contradiction assume that $x \in \left(X\setminus A\right)$, which is open by definition.\\
	Then, it exist $r>0$ such that $B_r\left(x\right)\subset\left(X\setminus A\right)$ and, equivalently, $B_r\left(x\right)\cap A = \varnothing$. \\
	By definition \vref{limit-in-metric-spaces} we can choose a point from the sequence which is arbitrarily close d to $x$, but for all $n$ we have $x_n \notin B_r\left(x\right)$, so we have a contradiction.
	
	\textit{Sufficient condition $(\impliedby)$:}\\
	By hypothesis we have that any sequence $\{x_n\}_{n \in \NN}$ converges to $x$ which is a point of $A$.	Choose $x \in \bar{A}$; we will prove that $x \in A$. \\
	Consider $B_{\frac 1 n}\left(x \right)$, by definition of adherence point we can choose a sequence $\{x_n\} \subset A$ such that:
	$$
		\forall n \ 
		\exists \, x_n 
		\in \left( A \cap B_{\frac 1 n}\left(x \right) \right)
	.
	$$
	Notice that $d\left(x_n,x\right)<\frac 1 n \to 0$, and so $x_n \to x$. Having choose $x_n \in A$ we have, by hypothesis, that $x \in A$.\\
	Thus $A = \bar{A}$ and $A$ is closed.
\end{proof}

Having considered ``all the possible sequences included in a set'', the following definition make sense:
\begin{defn}
	The \emph{sequence closure} of a set $X$ is: $$\{x \in X: \exists \, \{x_n\} \in X : x_n \to x \text{ in } (X,d)\}$$
\end{defn}

\paragraph{Bolzano--Weierstrass} Here we state a very useful result on which we will return to in next chapters.

\begin{theo}[Bolzano--Weierstrass] \label{bolzano-weierstrass-theo}
	Consider a metric space given by $\RR^N$ coupled with any metric.\\
	Any bounded sequence contains at least one convergent subsequence.
\end{theo}

This theorem states that when working on $\RR^N$, having defined any metric, if we have any bounded sequence $\{x_n\}_{n \in \NN}$ it is always possible extract a subset of its elements obtaining a subsequence $\{x_{n_h}\}_{h \in \NN}$ which is convergent in the metric space. Note the convention of indexes and subindexes.
