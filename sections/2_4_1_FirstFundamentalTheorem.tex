%!TEX root = ../main.tex
This chapters contains the most important results of the real analysis; here we focus on calculus of one-valued functions, then, in the next chapter, we will consider also multi-valued functions. Some of those results are well-known also in lower calculus courses, but here we will approach the problem from a more technical point of view.

%This chapter is a sort of revision on the theory of function of one value. We will see the generalization of some elemental results of calculus.\\
%One of our goal is to find a link between differentiation and integration.\\
%For example consider $a,b\in \RR$, $f\in \Cc([a,b])$, and $F(x) = \int_a^x f(t) \dt$. The theorem of calculus we all know states that $F$ is differentiable in $[a,b]$ and $F'(x)=f(x)$ for all $x \in [a,b]$. How can we extend this result also to Lebesgue integrable functions?

\subsubsection{First fundamental theorem of calculus}

We try to define a generic goal. Consider the measure space $([a,b],\Lc([a,b]),\lambda)$, take $f\in L^1([a,b],\Lc([a,b]),\lambda)$ and set 
$$
	F(x) 
	= \int_a^x f(t)\dlam
,
$$
which properties does $F$ have?

\paragraph{Lebesgue points} First, our focus is on discontinuity points. The first step is to define a new notion for ``continuous points''; on those we are confident that there are no issues with integration and differentiation. 
\begin{defn}
A point $x\in[a,b]$ is a \emph{Lebesgue point} for a function $f$ if there exists a representative $\tilde{f}$ of $f$ ($\tilde f = f$ a.e.) such that:
$$
	\lim\limits_{h\to 0}
	\frac 1 h 
	\int_x^{x+h} |\tilde{f}(t) - \tilde{f}(x)|\, \dt 
	= 0
,
$$
where $h \to 0^+$ if $x=a$ or $h \to 0^-$ if $x=b$.
\end{defn}
The integral uses the $\lambda$ measure, the integral variable $t$ has been written in $\dt$ for clarity.

Notice that $\tilde{f}$ is a representative of the equivalence class $[f]$, given by almost-everywhere equality; it's typical of $\Lc^1$ spaces. 

Lebesgue points do not present discontinuity and are a sort of ``continuous points''; for instance, jump points are not a Lebesgue point; check it by considering:
$$
	f(t) = 
	\begin{cases}
		\frac{t}{|t|} & \text{if } t\in [-1,1] \setminus \{0\} \\
		0 & \text{if } t=0
	.
	\end{cases}
$$

Having checked that jumps are not Lebesgue points, one can wonder if a continuity point is a Lebesgue point. We have the following result.

\begin{theo}
	If $x_0$ is a continuity point for $f$, then $x_0$ is a Lebesgue point for $f$.
\end{theo}

\begin{proof}
	By definition of continuity:
	$$
		\forall \eps >0 \ \ \exists \, \delta :\ \ | x-x_0| < \delta \implies | f( x) -f( x_0)| < \eps ,
	$$
	we evaluate the quantity:
	$$
	\left| \frac{1}{h}\int_{x_0}^{x_0+h}| f( x) -f( x_0)| dt\right| \leq \frac{1}{| h| } \eps | \lambda ( x_0 +h-x_0)| =\frac{|h|}{|h|} \eps =\eps ,\ \forall \eps >0.
	$$
\end{proof}

\begin{theo}\label{theo-lebesgue-points}
	Let $f\in L^1([a,b],\Lc([a,b]),\lambda)$. \\
	Then almost any point $x\in[a,b]$ is a Lebesgue point of $f$. \footnotemark{}
\end{theo}
\footnotetext{For further discussion and a proof, see: W. Rudin, Real and Complex Analysis, 1987, page 138, theorem 7.6.}


\paragraph{The theorem} We are able to state a first description of the relation between differentiation and integration.
\begin{theo}[First fundamental theorem of calculus]\label{theo-first-fundamental-calculus}
	Let $f\in L^1([a,b],\Lc([a,b]),\lambda)$.\\
	The integral function $F(x) = \int_a^x f(t)$ is differentiable a.e.\ and $F'=f$ a.e. in $[a, b]$.
\end{theo}
\begin{proof}
	Let $x\in[a,b]$ be a Lebesgue point of $f$ and $h \neq 0$ such that $x+h \in [a,b]$.
	Notice that we can rewrite the value of a function in the point as follow:
	$$
		f(x) 
		= \lim\limits_{h \to 0}
		\frac 1 h 
		\int_x^{x+h} f(t) \, \dt.$$
	Consider now the incremental quotient of the integral function $F$; we need it equal to function $f$. So we write:
	$$
		\frac{F(x+h)-F(x)}{h}-f(x) 
		= \frac 1 h \int_{x}^{x+h} (f(t)-f(x)) \,\dt
	,
	$$
	then, by triangular inequality (see \vref{prop-triang-ineq-integral}), we have:
	$$
		\left| \frac{F(x+h)-F(x)}{h}-f(x) \right| 
		\leq \frac{1}{|h|} \int_x^{x+h} |f(t)-f(x)| \,\dt
	.
	$$
	Taking the limit for $h \to 0$ or $h \to 0^+$, if $x=a$, or $h \to 0^-$, if $x=b$, we have: 
	$$
		 \frac{1}{|h|} \int_x^{x+h} |f(t)-f(x)| \,\dt
		 \to 0
	.
	$$
	We have proved the theorem for any Lebesgue point, through the previous theorem (\vref{theo-lebesgue-points}) we know that the set of non-Lebesgue points has measure zero; so the thesis is proven.
\end{proof}

But this theorem opens another question: is $F$ continuous in $[a,b]$?