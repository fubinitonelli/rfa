%!TEX root = ../main.tex
\subsubsection{Integrals and Fubini--Tonelli's theorems} We have build all the ingredients to define the abstract integral on a multidimensional space. Consider the measurable space $(X\times Y, \Mg \otimes \Ng, \mu \otimes \nu)$. In case $ E \in \Mg \otimes \Ng$ the \textit{reduction formula} for characteristic functions holds:
$$\int_{X\times Y} \Ind_E \,\de(\mu \otimes \nu)
= (\mu \otimes \nu)(E) = \int_X \nu (E_x) \dmu = \int_Y \mu (E_y) \dnu$$


This formula can be generalized: the following result shows that it can be extended to measurable functions. This results is made of two theorem which are different assumptions but the same final result, the \textit{iterated integral formula}.

As painful as it may be to state and comment this two theorems separately for an association named ``Fubini$\otimes$Tonelli'', this course requires to at least understand where each one comes from and why they work so well together.\footnote{Somehow like the pieces of Exodia.} 

\begin{theo}[Tonelli] %1885-1946
	Consider the measure space $(X\times Y,\, \Mg \otimes \Ng,\, \mu \otimes \nu)$, and let $f$ be a function $(\Mg \otimes \Ng)$-measurable and defined everywhere in $X\times Y$.\\
	If the function is non-negative, namely $0\leq f \leq +\infty$, then we have the \emph{iterated integral formula}:
	$$
	\int_{X\times Y} f \, \de (\mu \otimes \nu)
	= \int_X \left( \int_Y f^x \, \dnu \right) \, \dmu
	= \int_Y \left( \int_X f^y \, \dmu \right) \, \dnu.
	$$
\end{theo}
Observe that the first part of the iterated integral formula is actually an abstract integral, as it is defined on a measure.

For this next theorem we require also completeness for both measures.

\begin{theo}[Fubini]% 1879-1943
	Consider the measure space $(X\times Y,\, \Mg \otimes \Ng,\, \mu \otimes \nu)$, and let $f$ be a function $(\Mg \otimes \Ng)$-measurable and defined everywhere in $X\times Y$.\\
	If $f$ is such that $\int_{X\times Y} |f| \, \de (\mu \otimes \nu) < +\infty$, then we have: 
	\begin{alignat*}{2}
		f^x &\in L^1(Y, \Ng, \nu) \quad &&\text{for almost any } x \in X; \\
		f^y &\in L^1(X, \Mg, \mu) \quad &&\text{for almost any } y \in Y.
	\end{alignat*}
	Moreover:
	\begin{align*}
		x & \mapsto \int_Yf^x\dnu\in L^1(X,\Mg, \mu); \\
		y & \mapsto \int_Xf^y\dmu\in L^1(Y,\Ng, \nu).
	\end{align*}
	Finally, the iterated integral formula holds.
\end{theo}

%
%\begin{theo}[Fubini--Tonelli] \label{fub-ton}
%	Let $\mu$, $\nu$ $\sigma$-finite and complete respectively on $\Mg$, $\Ng$. \\
%	Consider the measure space $(X\times Y,\, \Mg \otimes \Ng,\, \mu \otimes \nu)$, and let $f$ be a function $(\Mg \otimes \Ng)$-measurable and defined everywhere in $X\times Y$.
%	
%	\emph{Tonelli} \\%1885-1946
%	If the function is non-negative, namely $0\leq f \leq +\infty$, then we have the \emph{iterated integral formula}:
%		$$
%			\int_{X\times Y} f \, \de (\mu \otimes \nu)
%			= \int_X \left( \int_Y f^x \, \dnu \right) \, \dmu
%			= \int_Y \left( \int_X f^y \, \dmu \right) \, \dnu.
%		$$
%	\emph{Fubini} \\ % 1879-1943
%	If $f$ is such that $\int_{X\times Y} |f| \, \de (\mu \otimes \nu) < +\infty$, then we have: 
%		\begin{alignat*}{2}
%			f^x &\in L^1(Y, \Ng, \nu) \quad &&\text{for almost any } x \in X; \\
%			f^y &\in L^1(X, \Mg, \mu) \quad &&\text{for almost any } y \in Y.
%		\end{alignat*}
%		Moreover:
%		\begin{align*}
%			x & \mapsto \int_Yf^x\dnu\in L^1(X,\Mg, \mu); \\
%			y & \mapsto \int_Xf^y\dmu\in L^1(Y,\Ng, \nu).
%		\end{align*}
%		Finally, the iterated integral formula holds.
%\end{theo}
%Observe that the first part of the iterated integral formula is actually an abstract integral, as it is defined on a measure.\\ The completeness is required only for the Fubini part.

\begin{coro}
	Let $\mu$, $\nu$ $\sigma$-finite and complete respectively on $\Mg$, $\Ng$. \\
	Consider the measure space $(X\times Y,\, \Mg \otimes \Ng,\, \mu \otimes \nu)$, and let $f$ be a function $(\Mg \otimes \Ng)$-measurable and defined everywhere in $X\times Y$.\\
	If, in addition, we have that:
	$$
	\int_X \left( \int_Y |f^x| \,\dnu \right) \,\dmu
	= \int_Y \left( \int_X |f^y| \,\dmu \right) \,\dnu
	.
	$$
	Then we have:
	$$\int_{X \times Y} |f| \,\de (\mu \otimes \nu) < +\infty$$
	and the iterated integral formula holds.
\end{coro}
\begin{proof}[Proof of the corollary]
	Consider $|f|$: using Tonelli's theorem, we deduce that:
	$$\int_{X \times Y} |f| \,\de (\mu \otimes \nu) < +\infty,$$
	so we can apply Fubini's theorem and obtain the iterated integral formula.
\end{proof}

Thanks to Fubini's theorem if we have $\sum_{i,j \in \NN}|a_{ij}| < +\infty$, then:
$$\sum_{i \in \NN} \left(\sum_{j \in \NN} a_{ij}\right) =\sum_{j \in \NN} \left(\sum_{i \in \NN} a_{ij}\right),$$
and the double series converges.\\
Set $X=Y=\NN$, $\Mg = \Ng = \Pc(\NN)$ $\mu= \nu= \mu_c$, and the measure $\mu_c \otimes \mu_c$ is defined with the convention $0 \cdot \infty = 0$.
	
\paragraph{Completion of a product measure} Consider two complete measure space $(X, \Mc, \mu)$ and $(Y, \Nc, \nu)$ with $\sigma$-finite measure. Then $(X\times Y, \Mg \otimes \Ng, \mu \otimes \nu)$ is not complete, but we can consider its completion:
$$(X\times Y, (\Mg \otimes \Ng)^*, (\mu \otimes \nu)^*).$$
Then Fubini--Tonelli's theorem can be relaxed:

\begin{theo}
	If $(X\times Y, (\Mg \otimes \Ng)^*, (\mu \otimes \nu)^*)$ is the completion of $(X\times Y, \Mg \otimes \Ng, \mu \otimes \nu)$ and $f: X \times Y \to \RR$, or $[0, \infty]$, is $(\Mg \otimes \Ng)^*$-measurable, then Fubini--Tonelli's theorem hold but the inner integrals and the iterated integral formula are defined almost everywhere.
\end{theo}

If we case of the Lebesgue measure we have:
\begin{prop}
	The completion of
	$$(\RR^N \times \RR^M, \; \Lc(\RR^N) \otimes \Lc(\RR^M),\; \lambda_N \otimes \lambda_M)$$
	is
	$$(\RR^{N+M}, \; \Lc \left( \RR^{N+M} \right), \; \lambda_{N+M}).$$
\end{prop}

Observe that $(\RR \times \RR, \; \Lc(\RR) \otimes \Lc(\RR),\; \lambda \otimes \lambda)$ is not complete.\\
Take $A \in \Lc(\RR)$ such that $\lambda(A)=0$ and $B \in \Lc(\RR)$ such that $\lambda(B)>0$.\\
Consider then a Vitali set $\Vc \subset B$. The rectangle $\Rc = A \times B$ is such that $(\lambda \otimes \lambda)(\Rc)=0$ but $A \times \Vc \subset \Rc$ and $(A \times \Vc)_x = \Vc \notin \Lc(\RR)$ so that $A \times \Vc \notin \Lc(\RR) \otimes \Lc(\RR)$. 

%\subsubsection{Change of variables in $\RR^N$} %this part was not treated in slides.
%
%The following result holds:
%\begin{theo}
%	Let $\Omega$, $\Omega' \subseteq \RR^N$ be homomorphic open sets, and let $\boldsymbol\Phi:\Omega' \to \Omega$ be a diffeomorphism (\textit{i.e.} an homomorphism of class $\Cc^1$).\\
%	Consider $E\subset \Omega, E' \subset \Omega'$ $\Lc$-measurable sets such that $\boldsymbol\Phi(E')=E$. Then:
%	$$f\in \Lc^1(E,\Lc(E), \mu) \iff ( f\circ \boldsymbol){\Phi} |J \boldsymbol{\Phi}| \in \Lc^1 (E', \Lc(E'), \lambda)$$
%	where $\frac{\dmu}{\dlam}=|\det J\boldsymbol{\Phi}|$. If so, then $\int_E \dmu=\int_{E'} (f \circ \boldsymbol{\Phi}) \frac{\dmu}{\dlam} \,\dlam$.
%\end{theo}

%What's the meaning of homomorphic? GG